\chapter{Hovedtekst}
I løbet af projektets første to faser, inception og elaboration, har der været en løbende udvikling af projektets planlægning og produkt. Denne udvikling er sket på baggrund af ny viden, som gruppen har til stræbet sig fra virksomhedsmøder, relateret undervisning og professionel vejledning. \\
En projektrammeplan er blevet brugt til at danne grundlaget for projektets planlægning. Den generelle projektrammeplan indeholdt fire afleveringer, samt grundlæggende milepæle, der skulle suppleres med gruppens egne mål. 
(Bilag til vores milepæle) \\
Første aflevering var gruppens projektforslag, som skulle bestå af de første ideer om problemstillingen i forhold til, hvordan den skulle formuleres, afgrænses og takles. (Se interne bilag \ref{in:projektforslag})\\
Anden aflevering var et Inceptionsdokument, som bestod af en grundlæggende analyse af projektet. Dette dokument skulle bl.a. indeholde systemets omfang, overordnede kravspecifikationer og prioritering af krav. (Se interne bilag \ref{in:inceptions})\\
Tredje aflevering bestod af det samlede arbejde fra første iteration af elaborationsfasen. Denne aflevering tog udgangspunkt i inceptionsdokumentet og startede med at opdatere projektets overordnede - og detaljerede kravsspecifikationer. Der blev udarbejdet statiske - og dynamiske analysediagrammer, samt statiske designdiagrammer. Ud fra disse planer, blev der udviklet et stykke software, som skulle være første skridt på vejen, mod projektets endelige kildekode.\\
Fjerde aflevering bestod af det samlede arbejde fra anden iteration af elaborationsfasen. Denne aflevering var en fortsættelse af første iteration, hvor der var fokus på at runde projektet af. Alle analyse- og designdiagrammer blev opdateret og der blev suppleret med et dynamisk designdiagram. Der var særligt fokus på at implementere den funktionalitet i kildekoden, som blev prioriteret højst, i forhold til fremtidig systemudvikling.\\
Det kommende afsnit vil gå i dybden med arbejdet og resultatet af det, som er blevet udført i første og anden iteration.

\section{Overordnet kravspecifikation (resume, opdateret)}
Figuren viser to forskellige systemafgrænsninger(Se figur \ref{fig:overkrav}). Den ene systemafgrænsning er af et login system som gruppen har valgt at simulere, da det originale Sensum har et login system tilknyttet, login modulet bliver i dette tilfælde brugt til at en given bruger kan logge ind i systemet og derud fra se sine sager. Den anden systemafgrænsning som er sagsforløb der det modul som gruppe har lagt fokus i, der er så her lagt størst fokus på opret sag og find sag, som en essentiel i opstarten af systemet, da en sagsbehandler skal kunne oprette en sag på en given borger, inden den videre behandling kan forsætte. Dertil er det under sagsudredningen af borgerne også essentiel for en sagsbehandler at kunne gå ind i systemet og nemt finde en borger der er oprette en sag på.\\
Den brugsmønster model som blev udarbejdet i løbet af inceptionsfasen, den vi har arbejdet ud fra igennem elaborationsfasen også, og den indeholder de brugsmønstre som gruppen har fundet nødvendige for at få netop modulet sagsudredning til at fungere som en kommunalansat skal bruge for at udrede inden for deres område. 
\begin{figure}[h]
  \includegraphics[scale = 0.4]{./PNG/krav/overkrav.PNG} 
  \caption{Gruppens overordnede krav for fuld størrelse se interne bilag afsnit \ref{sec:diverse} figure \ref{fig:fuldoverkrav}}
  \label{fig:overkrav}
\end{figure}

\section{Detaljeret kravspecifikation}
Ud fra den overordnede kravsmodel (Se figur \ref{fig:overkrav}) er der blevet udvalgt forskellige krav, som er set som mere kompliceret krav end resten af de overordnede krav. Der vil blive lagt vægt på to krav i dette afsnit. Disse krav er opret sag og find sag. For at se alle af de detaljeret brugsmønstre se interne bilag firgur \\
\textbf{Opret sag} \\
\begin{center} \label{tab:opretSag}
\begin{longtable}{|p{18cm}|}
\hline
\textbf{Brugsmønster: }Opret sag \\
\hline
\textbf{ID: }1 \\
\hline
\textbf{Primære aktører: }Sagsbehandler, afdelingsleder, administrativt personale\\
\hline
\textbf{Sekundære aktører: }CPR\\
\hline
\textbf{Kort beskrivelse: }En aktør kan oprette en sag, som gemmes i systemet. \\
\hline
\textbf{Prækonditioner: }Aktør skal være logget ind.\\
\hline
\textbf{Hovedhændelsesforløb: }\newline 
Starter når en borger henvender sig til kommunen \\
1. Aktør indtaster CPR nummer, borgers navn, begrundelse for henvendelse \\
2. Gemmer indtastet data.	
\\
\hline
\textbf{Postkonditioner: }sag oprettet\\
\hline
\textbf{Alternative hændelsesforløb: }\\
\hline
\end{longtable}
\end{center}
Som vist på figur \ref{tab:opretSag} er der ikke noget specielt svært ved dette brugsmønster. I gennem gang af dette kom der mange forskellige ting op, omkring hvad der ville være fornuftigt at skulle bruge til at oprette sagen. Det endte ud med at der skulle bruges en person, som sagen omhandle samt en grund til at sagen skulle eksistere. \\
\textbf{Find sag} \\
\begin{center} \label{tab:findSag}
\begin{longtable}{|p{18cm}|}
\hline
\textbf{Brugsmønster: }Find sag \\
\hline
\textbf{ID: }3\\
\hline
\textbf{Primære aktører: }Sagsbehandler\\
\hline
\textbf{Sekundære aktører: }none\\
\hline
\textbf{Kort beskrivelse: }Skal kunne søge i sager og få en sag vist\\
\hline
\textbf{Prækonditioner: }Der skal være oprettet en eller flere sager\\
\hline
\textbf{Hovedhændelsesforløb: }\\
Starter når den primære aktør skal finde en sag.\\
1. Skal søge en sag på sagsnummer, CPR nummer, eller navn.\\
2. Vise en liste over de sager som blev fundet.\\
3. Hvis en sag bliver valgt.\\
3.1 Vis den valgte sag.\\
\hline
\textbf{Postkonditioner: }\\
\hline
\textbf{Alternative hændelsesforløb: }\\
\hline
\end{longtable}
\end{center}
Brugsmønsteret find sag (\ref{tab:findSag}) er til for at finde en vilkårlig sag. Dette skal kunne ske ud fra sagsnummer fra sagen, navnet på personen sagen omhandler og deres CPR-nummer. Det er vigtigt at brugsmønsteret kan åbne en sag specifik sag, eller stoppes hvis ikke det er den rigtige søgning. \\
For se krav se interne bilag afsnit \ref{sec:diverse} figur \ref{fig:krav} og figur \ref{fig:ikkeKrav} 
\section{Første iteration}

Formålet med elaborationsfasen er at skabe en prototype af produktet MMMI(Se ordliste). Denne udvikling sker på baggrund af inceptionsfasen (Se interne bilag afsnit \ref{in:inceptions}), hvor der blev opstillet en række krav til projektet.  For at kunne nå de mål bliver der lavet nogle mindre iterationer herunder analyse, design, kode og test som udarbejdes, basseret på den overordnede kravspecifikation. 

\section{Analyse}
Der er blevet lavet en række diagrammer over de forskellige detaljeret brugsmønster.\\
Der er to forskellige typer af analyse diagrammer, der er de dynamiske og de statisk.

\subsection{Statiske side af analysen}
\begin{figure}
  \includegraphics[width=\linewidth]{./PNG/analyseKlasseDiagram.PNG} 
  \caption{Analyse klasse diagram.}
  \label{fig:AKlasse}
\end{figure}



\section{Design}

\subsubsection{Den statiske side af design}
\begin{figure}[h]
\includegraphics[width = \linewidth]{./PNG/design/fulddesignklassediagram.PNG}
\caption{Klassediagrammfuld. For fuld størrelse se interne bilag afsnit \ref{sec:diverse} figur \ref{fig:fuldDesignKlasseDiagram}}
\label{fig:desginklasse}
\end{figure}
Diagrammet i \ref{fig:desginklasse} beskriver de klasser vi har fundet, der skal implementeres i systemet. Hver klasse har attributter og en del skal have skrevet getter / setter-metoder, men for at diagrammet ikke skal blive for stort og uoverskueligt, valgte gruppen at undlade dem.\\
Diagrammet er blevet designet med tanker på en tydelig lagdeling og repræsentere systemets domænelag. Det blev besluttet at ”Department”-klassen skulle benyttes som facadeklasse til præsenta- tionslaget, mens datahandler interfacet bruges til kommunikation mellem domæne- og persistentslag.\\
\textbf{Department – Facade klasse}\\
Der var særligt fokus på attributterne: ”id” af typen int og ”caseMap” af typen Map\textless String, Case\textgreater . Det var tænkt at ID skulle bruges til at godkende adgangen til sager i databasen, for at imødekomme projektets ”Dataafgrænsning”. ID er af typen int, for at holde den simpel. Kravet til dataafgrænsningen er, at en ansat der henter information omhandlende en borger, ikke må blive gjort opmærksom på sager, der administreres af andre afdelinger end ansattes egen.\\
Tanken bag attributten ”caseMap”, var at department skulle have en liste af case-objekter som kunne bruges til at hente de relevante informationer frem for at sende dem til præsentationslaget. Der blev valgt at bruge et Map da Key/Value-strukturen gør det nemmere at finde en sag. Den Key der blev valgt til caseMap er caseNumber og skal bruges til metoden openCase(…).\\
Klassens metoder, som ”search(…)”, ”openCase(…)”, mm. er designet til at håndtere den logik, som controllerne fra præsentationslaget skal bruge. Metoderne repræsentere de funktioner som er ønsket i det grafiske userinterface (GUI).\\
Når metoden ”search(…)” kaldes, skal logikken først hente data fra persistenslaget. Dette sker gennem Interfacet PersistanceInterface som opretter et objekt af klassen ”SearchCase”, som er designet til at være en dataklasse i domænelaget, for data der er relevant til en brugers søgeord.\\ \\
\textbf{SearchCase – Dataklasse}\\
Klassen består af en constructor med attributterne: ”citizen”, ”caseNumber”, ”caseStatus”, ”date”, ”reason”, ”employeeName” og ”employeeID”. Disse informationer er ikke et udtryk for alt der ligger i en sag, men er nok til at præsentere en borgers sag. \\
Hvis der er et ønske om at hente den fulde sag, gøres dette ud fra ”caseNumber”-attributten gennem ”openCase(…)” metoden, fra ”Department”-klassen. Den data der er relateret til en sag, hentes fra persistenslaget, gennem interfacet ”PersistanceInterface”.\\
\\
\textbf{PersistanceInterface – Datahandler interface}\\
For at behandle dataflowet mellem persistenslaget og domænelaget, er der blevet designet et interface med metoderne: ”create()”, ”read()”, ”update()” og ”delete()”. Disse metoder blev valgt for at kunne oprette, hente, ændre og slette data, som opbevares i persistenslaget. For at domænelaget kan sende dataforespørgsler til persistenslaget, skal der implementeres klasser med det påkrævede interface i persistenslaget.\\ 
\\
\textbf{Case : Employee : Citizen – Objektklasser}\\
Objektklasserne blev designet med tanken om at præsentationslaget og persistenslaget, skulle arbejde med konkrete instanser af disse. Persistenslaget skulle tilføje den data, som var forbundet med objektet, som blev kaldt gennem domænelaget. Præsentationslaget skulle registrere indholdet i objekterne, pakke det ud og præsentere det som ønsket. \\
\\
\textbf{JobTitle – Rollebaseret kontrol}\\
Superklassen ”JobTitle” blev designet til at kontrollere en given medarbejders rettigheder i systemet. Der blev taget udgangspunkt i tre jobstillinger, som blev skrevet op som subklasser - sekretær, sagsbehandler og afdelingsleder. Når en medarbejderklasse blev oprettet, skulle det være et krav, at de fik tildelt en stilling (JobTitle). Denne stilling skulle kunne ændre sig i systemet, f.eks. på grund af en forfremmelse, uden at medarbejderen skulle registreres på ny i systemet.\\
Stillingen som en medarbejder blev tildelet, skulle begrænse funktionaliterne medarbejder har adgang til. 
Dette sikrede de kun havde adgang til de områder, som var nødvendig til at udfører deres opgave. 
En sekretær skulle f.eks. ikke kunne andet end at registrere en person, som søger behandling samt årsagen for at der søges. En sagsbehandler skulle have fuld adgang til relevante sager, dvs. kun sager der er oprettet i den afdeling sagsbehandler er tilknyttet. En afdelingsleder skulle kunne det samme som en sagsbehandler, men også kunne styre hvilke sager de forskellige sagsbehandlere skulle arbejde på.\\
 \\ \\ \\ \\
\textbf{Multiplicitet/Multiplicity – Klassernes forbindelser}\\
Forbindelserne mellem klasserne i designdiagrammet, består primært af aggregeringer. Klassen ”Employee” er forbundet med en many-to-one aggregering til klassen ”JobTitle”. Dette er blevet valgt, da objekter af klassen ”JobTitle” ikke skal blive slettet, hvis et ”Employee”-objekt fjernes. Tanken der ligger til grund for dette, er at ”JobTitle” kan være forbundet til flere forskellige ”Employee”-objekter. \\
Klassen ”Department” er forbundet til ”Employee” med to aggregeringer. Første aggregering er ”employeeList”, som er en one-to-many relation. Aggregeringen er blevet lavet, for at indikere at der kan være mange der arbejder i samme afdeling, men det skal stadig være muligt at slette en ”Department”, uden at slette ”Employee”-instanserne. Anden aggregering er ”departmentManager”, som er en one-to-one relation. Denne aggregering er blevet lavet, for at demonstrere at der altid skal være en leder i en afdeling. Igen skal det være muligt at slette en ”Department”, uden at det sletter ”Employee”.\\
Klassen ”Department” er forbundet til ”Case” med en enkelt aggregering. Dette er en one-to-zero-or-more forbindelse, som betyder at en afdeling ikke altid har nogen sager, men en sag er altid forbundet til en afdeling. Aggregeringen blev valgt, da en sag ikke forsvinder, hvis afdelingen bliver fjernet. \\
Klassen ”Case” er forbundet med to aggregeringer til klassen ”Citizen”, som også er forbundet med en komposition tilbage igen. Den første aggregering der er lavet, er en zero-or-many-to-zero-or-many relation, som omhandler en sags henvisning. En person kan blive henvist af nul eller flere personer, omhandlende det samme problem og på samme måde er det også muligt for en person at have henvendt sig for at få oprettet nul eller flere sager. Anden aggregering er lavet af samme grund som kompositionen er. En ”Citizen” kan have en til flere sager i gang på en gang, men en sag skal altid være angående en enkelt ”Citizen”. Aggregeringen er anvendt da det skal være muligt at fjerne en ”Case” fra systemet, uden at fjerne den ”Citizen” sagen omhandler. Kompositionen er derimod blevet brugt, da en ”Case” altid vedrører en ”Citizen”, og hvis denne fjernes fra systemet, skal de relaterede ”Case”-objekter fjernes, da de ikke længere vedrører nogen.\\
Klassen ”SearchCase” er forbundet med en komposition fra interfacet ”PersistanceInterface”. Da et interface ikke kan instantieres, har det ikke nogen multiplicitet. Kompositionen er blevet valgt, da ”SearchCase” ikke fungere uden interfacet, og forsvinder derfor, hvis interfacet fjernes.

\subsubsection{Den dynamiske side af design}
Der blev ikke arbejdet med den dynasike side af design, da teorien ikke var blevet introduceret i første iteration. 

\subsection{Implementering}
Målet med første iteration var at skabe et domænelag, der havde de funktionaliteter, som var nødvendige for at systemet kunne fungere. Dette blev gjort så gruppen havde en bedre forståelse for hvordan systemet skulle se ud. 
Gennem første iteration blev der lavet to klasse (case og department) der implementerede begge brugsmønstre. ”Case” håndterede hvilke data der skulle gemmes i en sag, mens ”department” holdt styr på hvilke specifikke sager der var og hvordan de skulle tilgås.  Nedenfor ses de forskellige attributter i ”Case” klassen. (Se figur \ref{kode:caseatt})
\begin{figure}[h]
\begin{lstlisting}
public class Case {
    private static int socialCount = 1;
    private static int handicapCount = 1;

    private final String caseNumber;
    private String caseStatus;

    private final String date;
    private final Citizen regardingCitizen;
    private final List<Citizen> requestingCitizen = new ArrayList<>();
    private final Map<String, String> information = new HashMap<>();
    private final int departmentID;
\end{lstlisting}
\caption{Case klassens attributter}
\label{kode:caseatt}
\end{figure}\\
Der bruges et map til at holde den data der ændre og skrives ind.\\ 
Klassen indeholder en status på om sagen er lavet, i gang eller færdig.\\ 
Der kan kun være én person sagen omhandler, så der er en instans af klassen. Der kan være flere der henvender sig angående den samme person. Dette blev håndteret ved at lave list over disse personer. 
\begin{figure}
\begin{lstlisting}
public List<Case> search(String searchWord) {
	List<Case> listCases = new ArrayList<>();
	
	for (Case searchCase : caseMap.values()) {
		if (searchCase.getRegardingCitizen().getCprNumber().equalsIgnoreCase(searchWord)
				|| searchCase.getCaseNumber().equalsIgnoreCase(String.valueOf(searchWord))
				|| searchCase.getRegardingCitizen().getName().equalsIgnoreCase(searchWord)) {
            listCases.add(searchCase);
		}
	}
	return listCases;
}
\end{lstlisting}
\caption{Department search, finder alle sager der har søgeordet i}
\label{kode:search}
\end{figure}
Den data der kan vises i præsentationslaget kaldes igennem facaden "Department". Det var også her at alle de forskellige sager blev gemt i denne iteration. Der bliver genereret et sagsnummer hvor det er muligt at se hvilken afdeling sagen tilhører. \\ 
I figur \ref{kode:search} kan det ses hvordan klassen ”Department” finder en sag. Den skal bruge en ”string” som kunne være CPR-nummer eller navn for personen den omhandler eller  sagsnummeret. \\
Som det kan ses i koden er bliver der ikke returneret en sag, men en liste af sager.\\
Dette valgte gruppen at gøre fordi der kan søges på et navn, som ikke er unikt og der kan derfor findes mere end en persons sager. Samtidig er det også muligt at en person har mere end en sag, så selv hvis der kun er én person med et givent navn eller der søges på et CPR-nr kan der stadig returneres mere end én sag. Dette nåede ikke at blive implementeret i denne iteration. 



\subsection{Test}
\begin{figure}[hbt!]
\begin{lstlisting}
public static void main(String[] args) {
	while (!quit) {
                rollInfor();
                text = sc.nextLine().toLowerCase();

                switch (text) {
                    case "secretary":
                        Employee trine = detDepartment.createEmployee("trine", 
                        252525, "secretary");
                        commands(trine);
                        break;
                    case "caseworker":
                        Employee mads = detDepartment.createEmployee("mads", 
                        353535, "caseworker");
                        commands(mads);
                        break;
                    case "departmentmanager":
                        Employee martin = detDepartment.createEmployee("martin", 
                        262626, "departmentmanager");
                        commands(martin);
                        break;
                    case "quit":
                        System.out.println("Du quiter");
                        quit = true;
                        break;
                    default:
                        System.out.println(text + "Er ikke en kommando\n");
                        break;
                }
\end{lstlisting}
\caption{tekst under figuren}
\label{kode:1main}
\end{figure}
Der blev ikke lavet unit test på første iteration, da det ikke blev set som en nødvendighed. Det der blev fokuseret på var systemtest som skulle sikre at forståelsen med projektet og hvordan de forskellige brugsmønstre ville fungere i et færdigt system. Der blev fundet flere fejl/mangler igennem systemet. Vist på koden (Se figur \ref{kode:1main}) kan det ses at der bliver lavet en ny ”employee” hvilket skyldes det ikke var lavet så den kunne huske mere end en ad gangen. Det kan ses I figur \ref{kode:1commando} er det de forskellige kommandoer som kan bruges returneres det vil derefter testes om personen der er logget ind. \\
Dette blev den eneste rigtige test da dette kom ud på at få ideen til hvordan systemet skulle fungere helt på plads så databasen og præsentations laget kunne blive lagt på. 
\begin{figure}[hbt!]
\begin{lstlisting}
public static String loop() {
        while (true) {
            Scanner s = new Scanner(System.in);
            String t = s.nextLine().toLowerCase();

            switch (t) {
                case "back":
                    return t;
                case "create case":
                    return t;
                case "close case":
                    return t;
                case "add information":
                    return t;
                case "assign case":
                    return t;
                case "reassign case":
                    return t;
                default:
                    System.out.println(t + " Er ikke en kommando \n");
            }

        }
    }
\end{lstlisting}
\caption{loopet der ser om kommandoen kan er rigtig}
\label{kode:1commando}
\end{figure}
Dette blev den eneste rigtige test da dette kom ud på at få ideen til hvordan systemet skulle fungere helt på plads så databasen og præsentations laget kunne blive lagt på.
\newpage


\section{Anden iteration}

\subsection{Analyse}

\subsubsection{Statisk analyse}
\begin{figure}[htb!]
  \includegraphics[scale = 0.7]{./PNG/analyse/analyseklassediagramOpdateret.PNG} 
  \caption{Analyse klasse diagram opdateret til anden iteration}
  \label{fig:2analyseklasse}
\end{figure}
Da der har været fokus på at få integreret en database og en funktionel grafisk brugergrænseflade i 2. iteration, består de primære ændringer i den statiske analyse, af arbejdet med at revidere klassediagrammet, med henblik på at tilføje persistens og præsentation. Der er bl.a. blevet ændret på hvordan de tre klasser der repræsenterer ansatte, og klassen der repræsenterer borgere, beskrives. \\
Der blev valgt at gå væk fra en overordnet abstrakt klasse (Se figur \ref{fig:AKlasse})kaldet ”Person”, som ”Borger”, ”Afdelingsleder”, ”Sekretær” og ”Sagsbehandler” klasserne arvede fra. En abstrakt klasse kaldet ”medarbejder” blev lavet, som ”Afdelingsleder”, ”Sekretær” og ”Sagsbehandler” arvede fra i stedet, og holdt ”Borger” som en separat klasse. Dette betød at ”Borger” fik navn og CPR-nr. som attributter, til forskel fra 1. iteration hvor den kun havde CPR-nr. og arvede navn fra ”Person”. Yderligere blev medarbejderID flyttet til superklassen, da alle arvende klasser skulle bruge den. \\
Det blev droppet at beholde ”Afgørelsesbrev” som klasse, da det gav mere mening at lægge den i præsentationslagets klasse ”Formular”, som også repræsenterer alle andre formularer der beskrives i VUM.\\
Klassen ”Sag” er blevet rykket ned til datalaget og kan tilgås gennem Databehandler klassen, som er den klasse der repræsenterer datalaget og skal varetage al kommunikation med databasen.\\
De vigtigste metoder og attributter, på diverse klasser, er blevet opdateret i forhold til den nye viden, der er blevet indsamlet gennem 2. iteration. Der er blevet lavet nye metoder er på klassen (Se figur \ref{fig:2analyseklasse})”MMMI Brugerflade”, som er ”findSag”, ”opretSag” og ”gemSag”. På klassen ”Afdeling” er metoderne ”findSag” og ”gemSag” blevet lavet og på klassen ”Databehandler” er ”findSag” og ”skrivSag” blevet lavet. De nye attributter som er tilføjet, er ”begrundelse”, ”borgerID”, ”oprettetDato” og ”sidstRedigeretDato” på ”Sag”. \\
Den nye klasse ”SøgsResultat”, bruges til at holde data samlet i forbindelse med en given søgning af en sag. De attributter som vises i klassen ”SøgsResultat”, repræsenterer en begrænset delmængde af hvad der gemmes når en sag oprettes, men det er nok til at en bruger vil være i stand til at identificere en given sag for at åbne den.\\
Klassen Database repræsenterer den postgresql database er blevet implementeret i iteration 2.

\subsubsection{Dynamisk analyse}
I anden iteration er der blevet udarbejdet operationssekvensdiagrammer i forhold til ”opretSag()”, ”opdaterSag()” og ”find()”. \\
Operationen ”opretSag()” (Se afsnit \underline{\ref{opret} opretSag}) og ”opdaterSag()” (Se afsnit \underline{\ref{opdaterSag} opdaterSag}) er operationer der stammer fra opretSag fra første iteration. I de pågældende sektioner vil der beskrives de overvejelser der blevet fortaget i forhold til de to operationer samt beskrivelse og vurdering. \\
Operationen find() er den operation der stammer fra ”findSag()” fra første iteration. I afsnit \underline{\ref{find} find} beskrives der overvejelserne om f.eks. fra ”findSag()” til find() og beskrivelse samt vurdering af operationen.\\ 
\textbf{opretSag():} \label{opret} \\ 
Operationen ”opretSag()” er den funktion der opretter en sag i systemet. Der gjort overvejelser om at håndtere opdaterSag() i samme funktion som ”opretSag()”, men det ville ikke give mening samtidig med at det ville gøre diagrammet uoverskueligt. Derfor er de adskilt i to operationer. \\
\begin{figure}[htb!]
  \includegraphics[scale = 0.5]{./PNG/analyse/opretsag2.PNG} 
  \caption{Sekvensdiagram for opret sag opdateret til anden iteration Se interne bilag \ref{sec:diverse} figur \ref{fig:2fopretsag}}
  \label{fig:2opret}
\end{figure}
I figur \ref{fig:2opret}. begynder funktionen ved at en sagsbehandler eller sekretær vælger ”opretSag()”. Brugergrænsefladen returner en sagsåbningsformular, som skal udfyldes. Når sagsbehandleren har indtastet sagsoplysningerne vælger sagsbehandler at gemme sagen. Operationen ”gemSag()” sender besked til Afdeling som derefter benytter operationen ”skrivSag()” til Databehandleren og til sidst sender Databehandleren beskeden videre til databasen. Databasen returner en bekræftelse tilbage til Databehandleren som returner den igennem Afdeling og til sidst fortæller aktøren at sagen blev oprettet. \\
Ved at anvende ”opretSag()” operationen har en sagsbehandler eller sekretær mulighed for at vælge en formular og udefra den gemme den og dermed oprette en ny sag. Oprettelsen sker når borgeren henvender sig og dermed kan sekretæren eller sagsbehandleren oprette en sag på den pågældende borger. For at kunne håndtere om en borger allerede har en sag eller om der findes oplysninger om borgen, benyttes ”find()” operation, som kan læses i afsnit \underline{\ref{find} find}\\
\textbf{opdaterSag():} \label{opdaterSag}\\
Operationen ”opdaterSag()” med en kombination af ”find()” operationen gør det muligt for en aktør at kunne finde en specifik sag og dermed åbne den og opdatere den. \\
\begin{figure}[htb!]
  \includegraphics[scale = 0.62]{./PNG/analyse/opdaterSag.PNG} 
  \caption{Sekvensdiagram for opdater sag opdateret til anden iteration}
  \label{fig:2opdater}
\end{figure}
I figur \ref{fig:2opdater} ses det at en sagsbehandler eller sekretæren vælger at fremsøge en sag. ”findSag()” beskeden bliver sendt igennem afdelingen og databehandleren. Databehandleren sørger for at sende beskeden til databasen og fremskaffe den søgte sag. Denne returneres til databehandleren og der bliver oprettet et sagsobjekt. Dette sagsobjekt returneres til afdelingen og herefter til brugerfladen og til sidst bliver den vist til brugeren. Når brugeren vælger at ”indtastNyData()” til det pågældende sagsobjekt, med de pågældende sagsoplysninger sendes til afdelingen via operationen ”sendData()”. Afdelingen sørger for at opdater sagsobjektet igennem beskeden ”opdaterSagsobjekt()” hvor sag returner sagsobjekt med sagsoplysningerne og det sagsobjekt og data sendes til Databehandleren igennem beskeden ”sendData()” som derefter sender besked til databasen om at gemme de nye oplysninger. Databasen returner en bekræftelse igennem databehandleren, afdelingen og til sidst sendes det til brugerfladen som så bekræfter overfor aktøren at sagen blev opdateret. \\
Det vurderes at denne operation er en væsentlig del af sagsforløbet og skulle kunne muliggøre for en sagsbehandler eller sekretær at kunne opdatere en sag ved at kombinere operationen med at finde en sag først og derefter opdatere den. \\
\textbf{find():} \label{find}\\
\begin{figure}[htb!]
  \includegraphics[scale = 0.56]{./PNG/analyse/find.PNG} 
  \caption{Sekvensdiagram for find sag opdateret til anden iteration}
  \label{fig:2find}
\end{figure}
Tanken med ”findSag()” i første iteration afsnit … har været at kunne fremsøge en pågældende sag frem. Der har været overvejelser om at kunne fremsøge en borger og en sag, fra en enkelt operation. Derfor blev ”findSag()” operationen lavet om til en mere general operation kaldt ”find()”.\\
I figur \ref{fig:2find} vælger en sagsbehandler eller sekretær at finde en sag eller borger baseret på et søgeord. Operationen ”find()” sender besked til grænsefladen. Baseret på den pågældende søgning, hvis det er en borger eller en sag der søges på, sendes ”find()” besked videre fra Afdelingen til Databehandleren der sørger for at sende besked til databasen, hvor der efterfølgende sendes svar fra databasen til Databehandleren der så baseret om det er en sag eller borger der er søgt på, opretter et SøgsResultat objekt, som Databehandleren sender tilbage. Afdelingen sørger for at sende beskeden til brugerfladen som til sidst viser søges resultatet til aktøren. \\
At kunne finde en sag frem eller en specifik borger er en vigtig funktion og dermed en vigtig operation for en sagsbehandler eller sekretær. Denne operation hjælper med at kunne finde sag på en pågældende borger og dermed hjælpe mere effektivt, ved at kunne hente alle sagsoplysninger. \\ 


\newpage
\subsection{Design}
Designdiagrammet i projektets anden iteration, blev opdateret med flere klasser og en tydelig lagdeling. En klar adskillelse af præsentations-, domæne- og persistenslaget er blevet illustreret og indeholder også subsystemernes klasser, med attributter og metoder. Det er valgt at getter og setter metoder ikke vises i diagrammet, da det ville gøre det meget større og sværere at overskue. (Reference til bilaget, med det fulde designdiagram)\\
For at få et samlet overblik se interne bilag \ref{sec:diverse} figur \ref{fig:opklassemedlog} og figur \ref{fig:opklasse}\\
\textbf{Præsentationslaget}
\begin{figure}[htb!]
  \includegraphics[width = \linewidth]{./PNG/design/UIopdateretKlassediagram.PNG} 
  \caption{Der blev valgt at undlade beskrivelsen af kontrollernes indhold og FXML-dokumenterne i subsystemet UI (Præsentationslaget), for at gøre diagrammet mere overskueligt. Præsentationslaget er blevet illustreret med pakker, som indeholder kontrollere og FXML-dokumenter for specifikke funktioner, samt klasserne som styrer systemets GUI.}
  \label{fig:2pre}
\end{figure}
\\\textbf{MMMI – Main Class}\\
Programmet køres gennem MMMI klassen, da det er programmets main class.\\
\textbf{MainController – Præsentations controlleren}\\
Klassen ”MainController”, er systemets primære FXML-handler. Den blev lavet så det var muligt at skifte funktionalitet, på baggrund af et brugerinput, i samme root-node.
\textbf{RunFxmlLoader – Præsentationsmetoden}\\
Klassen ”RunFxmlLoader”, indeholder funktionaliteten som bruges til at skifte FXML-dokumenter. Den bliver arvet af ”MainController”, og implementere, som det eneste, metoden der bruges til at skifte ”pane”, i systemet.\\
\textbf{CreateCase – Dokumentstyring og præsentation}\\
Subsystemet ”CreateCase”, blev designet til at begynde oprettelsen af nye sager i systemet og styre præsentationen af sagsdokumenter, samt brugerinteraktionen med disse. I slutningen af anden iteration var ”CreateCase”, pakket med kontrollerne ”CreateCaseController”, ”CaseOpeningController” og ”CaseInvestigationController”, samt FXML-pakken ”fxml”, som var pakket med de tilsvarende FXML-dokumenter.\\ 
Controlleren ”CreateCaseController”, blev designet til at styre bevægelsen mellem sagsdokumenter, samt at gemme brugerindtastet information fra disse dokumenter. Navngivningen kunne virke misvisende, i forhold til at controlleren bruges til at skifte mellem dokumenter, men peger på funktionaliteten bag gem funktionen. Når en given bruger benytter gem funktionen i en sag, sender controlleren de brugerindtastede informationer til domænelaget, for at få det registreret i systemdatabasen. \\
Hvis en sag gemmes for første gang i systemet, vil informationen som sendes gennem systemlagene, oprette en ny sag i databasen, som derefter modtager informationen. Controlleren er blevet navngivet efter denne proces, men har i sig selv kun at gøre med brugerinput og kommunikationen med domænelaget.\\
Controlleren ”CaseOpeningController”, er designet til at pakke informationerne indtastet i FXML-dokumentet ”caseOpeningNew”, og derefter sende det videre til ”CreateCaseController”, når en given bruger benytter gem funktionen.\\
Controlleren ”CaseInvestigationController”, har at gøre med informationen fra FXML-dokumentet ”caseInvestigation”. Controlleren er designet ligesom ”CaseOpeningController”, da de begge er designet til at håndtere information fra FXML-dokumenter, som er designet efter dokumenterne fra VUM. (kilde) \\
Designet af ”CreateCase”, var planlagt med fremtidig udvikling i tankerne, for at lave og implementere de dokumenter fra VUM, som mangler.\\
\textbf{FindCase – Sagsrelateret søgning} \\
Subsystemet ”FindCase”, er designet til at håndtere et brugerinput, i form af en søgning, og præsentere information relateret til søgeordet. Subsystemet består af klassen ”SearchResult” og controlleren ”FindCaseController”, samt FXML-dokumentet ”findCase”.\\
Klassen ”SearchResult”, er en placeholder som fyldes med den data, der hentes på baggrund af en given brugers søgeord. Controlleren ”FindCaseController”, henter informationen som ”SearchResult”, består af og præsentere dem gennem FXML-dokumentet ”findCase”.\\
\textbf{Home – Systemets startside} \\
Subsystemet ”Home”, er designet til at håndtere systemets startside. Subsystemet består af klassen ”UIEmployee”, som er en placeholder klasse for medarbejderen der er logget ind i systemet og controlleren ”HomeController”, samt FXML-dokumentet ”home”.\\
Klassen ”UIEmployee”, er en placeholder klasse, som fyldes med information relateret til medarbejderen der er logget ind i systemet. Controllerklassen ”HomeController”, bruger informationen fra ”UIEmployee”, til at finde de relevante informationer som vedrør medarbejderen og præsentere dem gennem ”home”-dokumentet.\\
\textbf{Domænelaget}
\begin{figure}[htb!]
  \includegraphics[width = \linewidth]{./PNG/design/domaeneOpdateretKlassediagram.PNG} 
  \caption{Domænelaget er blevet designet med to klasser, med attributter og metoder, og et interface.}
  \label{fig:2dom}
\end{figure}
\\\textbf{IDomain – Provided interface}\\
IDomain interfacet bruges til at vise præsentationslaget hvilke metoder der er tilgængelige i domænelaget, hvad de returnerer samt hvilke attributter der er nødvendige for at benytte metoderne. Interfacets metoder bliver implementeret i ”Department” klassen.\\
\textbf{Department – Facadeklasse}\\
Klassen ”Department” har en private constructor “Department()”, attributterne ”departmentID”, ”dataHandler”, ”mmmiEmployee”, ”loginEmployee” og ”departmentInstance”, samt metoderne ”getInstance()”, ”search(…)”, ”openCase(…)”, ”saveCase(…)”, ”isCitizen(…)”, ”getNote(…)”, ”writeNote(…)”, “getEmployee()” og “receiveEmployee()”.\\
For at imødekomme projektets underspørgsmål ”Hvordan sikres det at aktøren kun har adgang til det aktøren har brug for?” og dataafgrænsningen fra sagsudredning i semestercasen, er der blevet lagt stor vægt på attributten “departmentID”. \\
Dette ID tildeles via en setter metode, ”setDepartmentID(…)”, når en medarbejder logger ind i systemet. Singleton designpattern blev valgt, da der er nødvendigt at være sikre på at det er det samme ”Department”-objekt der arbejdes med gennem hele systemet.\\
\textbf{Employee – Dataklasse, repræsenterer den medarbejder der er logget ind}\\
Klassen ”Employee” er designet til at håndtere tjek af rettigheder i forhold til forskellige medarbejderroller. Når en bruger logger ind i systemet, sendes der en employeeID til ”Department” som kalder persistenslaget for at få data om medarbejderen for at oprette en instans af Employee.\\
Dette gøres for at det kan være muligt at der er styr på hvem der er logget ind, samt have mulighed for at tjekke rettighederne som medarbejderen har gennem metoden ”checkRight(…)”.\\
\textbf{Persistenslaget}
\begin{figure}[htb!]
  \includegraphics[width = \linewidth]{./PNG/design/datalagKlassediagram.PNG} 
  \caption{Persistenslaget er blevet designet med seks klasser og et interface.}
  \label{fig:2dataklassediagram}
\end{figure}
\\ 
\textbf{IDataHandler – Provided interface} \\
Interfacet ”IDataHandler”, bruges til at tillade og etablere kontakten mellem domænelaget og persistenslaget. Metoderne som interfacet består af implementeres i ”DataHandler” klassen og kaldes i domænelaget, når databasen skal kontaktes.\\
\textbf{DataHandler – Facadeklasse}\\
Klassen ”DataHandler”, er blevet designet til at håndtere dataforespørgsler sendt fra domænelaget og behandle dem som ønsket i systemets database. Klassen implementerer ”IDataHandler” interfacet og metoderne der følger med, som specificeres så data bliver hentet og/eller gemt, når de bliver kaldt fra domænet.\\
Når der skal hentes data fra databasen, er metoderne ”readCase(…)”, ”readCitizen(...)”, ”readEmployee(…)”, ”readAlternativeNotes(…)” og ”search(…)”, blevet designet for at kunne tilgå specifikke koloner, så den relevante information kan blive hentet. \\
Designet er lavet så at, når der skal gemmes data i databasen, benyttes metoderne ”writeCase(…)”, ”writeCitizen(…)”, ”updateCitizen(…)” og ”writeAlternaticeNote(…)”. Metodekaldet der skal bruges, afhænger af hvilken information der ønskes gemt. \\
Klassens associationer består primært af aggregeringer, med multipliciteten one-to-zero-or-many. ”DataHandler” er forbundet med aggregeringer til klasserne ”SearchCase”, ”Case”, ”Citizen” og ”Employee”. \\
\textbf{SearchCase – Placeholder til data fra “search(…)”}\\
Klassen ”SearchCase” indeholder udelukkende attributterne ”citizenID”, ”citizenName”, ”caseID”, ”caseStatus”, ”currentCaseDate”, ”createdDateCase”, “reason”, “employeeID”, “employeeName”.  Klassen bruges når der søges i databasen med metoden ”search(…)”, til at indeholde de data der er relevant at vise brugeren, i forhold til at kunne vælge den rigtige sag at åbne.\\
\textbf{Employee – Dataklasse, repræsenterer en medarbejder og alle dennes tilknyttede sager i afdeling}\\
Klassen ”Employee” indeholder udelukkende attributterne ”employeeID”, ”name”, ”roleID”, ”employeeCases” og ”rights”. \\
\textbf{Case – Dataklasse, repræsenterer en sag}\\
Klassen ”Case” indeholder attributterne “caseID”, “caseStatus”, “regardingCitizen”, “departmentID” som alle skal bruges for at oprette en ny sag i tabellen ”case” i databasen. Yderligere er der en attribut ”requestingCitizen”, som skal bruges for at forbinde en sag med en eventuel henvendende borger, samt en attribut ”caseContent”, som indeholder alle oplysninger som en sagsbehandler kan indtaste. \\
Der er også en constructor som sætter alle attributterne, samt en metode der bruges til at konstruere den query, som sørger for at ”caseContent”, bliver gemt i databasen.\\
\textbf{Citizen – Dataklasse, repræsenterer en borger og alle dennes oprettede sager i afdeling}\\
Klassen ”Citizen” indeholder attributterne ”citizenID”, ”firstName”, ”lastName”, ”cprNo”, ”streetName”, ”houseNo”, ”floor”, ”floorDirection”, ”regardingCitizen”, ”requestingCitizen”, ”Cases”. Den har 2 constructore, en som har de primære attributter der skal bruges for at oprette en Citizen i database, og en anden som har alle attributter.\\
\textbf{DataConnection – Håndtering af forbindelse til database}\\
Klassen ”DataConnection” blev implementeret for at vi kunne have de nødvendige informationer til at forbinde med databasen placeret et sted som kunne genbruges i andre moduler. \\
\subsubsection{Dynamisk design}
\textbf{Search}
Når en sagsbehandler eller leder skal søge igennem databasen for at finde en specifik sag. \\
I det en aktør klikker på søg knappen, bliver der lavet en efterspørgsels ud fra en specifik søge nøgle i form af String send fra presistenslageret til Departpartment i domainlageret. Denne string bliver sendt til Datahandleren. Hvor metoderne search bebrejder stringen efter om det er en borger eller en case der søges på. Afhængig af hvilken form af søgn, vil der blive sende en anmodning til database, hvor da validere om den specifikke søgning eksister. Vis den eksister sender databasen et resultset tilbage med de informationer der er ud fra den specifikke case. Den data bliver omdannes til nye objekter af ScearchCase.\\
disse objecter lægges i en liste af objecter SceartchCase. Herefter bliver det send tilbage til Department, Hvor dataene bliver delt op i forskellige Kategorier og lægges i et map ”ResultsetMap”. ResultsetMap sende er efter op til UI. I UI bliver ResultsetMap omdannet til objectet SearchResultset. Som bliver vis til aktøren. \\
Search er bygget efter aktøren behov for at finde en specifik sag på en borger. Hvor der skal være mulighed for at ændre eller vider bygge på en sag om en borgen. \\
VI har valgt at lader datalageret håndtere ansvaret for validere søgningen eftersom det er den der er den stærkeste tilknytning til databasen og for at reducere den gennerale kobling i projektet, hvilket der er lagt meget Vægt på i projektet.\\
\textbf{saveCase}
Når en aktør ar udfyldt noget af sagsoplysningerne i en sag. Har aktøren muligheden for at gemme arbejdet i databasen ved at klikke på gem. \\
Når en aktør klikker på gem sendes der et Map af strings ned fra presistenslageret til Departpartment.\\
I Department bliver der valideret om sagsindholdet. Map kan indeholde et caseID der er ligmed -1. vis dette er tilfældet vil der blive kaldt metode CreateCase i IDatahandler. Methodes formål er at lave et nyt object af Case med den information der er givet, herefter sendes tilbage til Department. \\
Vis caseID ikke er -1 vil er ikke blive lavet et nyt objekt. Da dataene ligger i et eksisterer objekt af case Her vil updateCase bliver kaldt. Updatecase’s formål er at opdatere objektet med den information der er giver. \\
Department sender nu ned til Datahandleren gennem interfaceset IDatahandler ved brugen af metoden WriteCase. WriteCase valider igen om caseID = -1. vis det er -1 vil der laves en peparesStatment der sender casens data til databasen. Og den sender en true op igennem systemet.


\subsection{Implementering}
\subsubsection{Nodefinder}
\textbf{Problem}\\
Hvordan kan systemme selv gemme alt data i et map.\\
\textbf{Eksemple } \\
Håndtering af input fra aktøren i UI. 
\begin{figure} [hbt!]
\begin{lstlisting}
public ArrayList<Node> getAllNodesFromRoot(Parent root) {
	ArrayList<Node> nodes = new ArrayList<>();
	addAllDescendentsFromNode(root, nodes);
	return nodes;
}
    
private void addAllDescendentsFromNode(Parent parent, ArrayList<Node> nodes){
	for (Node node : parent.getChildrenUnmodifiable()) {
		nodes.add(node);
		if (node instanceof Parent) {
			if (node instanceof TextInputControl) {
				if (!((TextInputControl) node).getText().isEmpty()) {
					String key = node.getId().substring(5);
					nodeMap.put(key, ((TextInputControl) node).getText());
				}
			}
			if (node instanceof RadioButton) {
				if (((Toggle) node).isSelected()) {
					String key = node.getId().substring(5,((RadioButton) node).getId().length() - 1);
					nodeMap.put(key, ((Labeled) node).getText());
				}
			}
			addAllDescendentsFromNode((Parent) node, nodes);
		}
    }
}
\end{lstlisting}
\caption{Data input håndtering}
\label{kode:nodes}
\end{figure}
\textbf{Løsning }\\
”NodeFinder” der har til formål at håndtering input Aktøren. Klassen har ansvaret igennem metoderne ”addAllDescendentsFromNode” og ”getAllNodesFromRoot” at samle alle nodes og gennem værdierne i et ”map” hvor nøglen er node id og værdien er den værdig der ligger i den node. Den har også formålet at sortere node der er i den forældrenode så vi kun kan få de nods der har en betydning den data vi skal has sat ind. \\
”getAllNodesFromRoot” tager alle "nods der ligger på den root node og ligger dem i en ArrayList. ”addAllDescendentsFromNode” tager alle node der er i arraylisten og køre dem igennem. Alle de nodes der er en parent node til en node køres igennem igen for at finde alle de nodes der enten er en ”Textinputcontrole” eller en ”radiobutton” og samler dem i et map.  Dette map bliver samlet med 2 andre maps ” ”cRegardingMap” og ”cRequestingMap” disse 3 maps udgøre til sammen hele sagen.\\
\textbf{Evaluering } \\
\begin{figure} [htb!]
	\includegraphics[scale = 0.7]{./PNG/imp/nodes.PNG} 
	\caption{Viser oprationen for hvordan den henter alle nodes}
	\label{fig:nodes}
\end{figure}
”NodeFinder” kan implementeres i alle fxml og samle den specifik data, hvis node id passer med det kolonnen navn i databasen. Dette gør at det bliver nemmer for fremtidige programøre at implementere fxml dokumenter i systemet og samle den data der bliver sat ind.

\subsubsection{Search case}
\textbf{Problem} \\
Hvilke informationer har domænelaget brug for og hvor skal det hentes fra, når der bliver foretaget en søgning?\\
\textbf{Eksempel} \\
Domænelaget sender to værdier til persistens laget. Værdierne skal behandles så "search-Key" bestemmer hvordan der skal søges og værdien "searchValue" hvad der skal søges efter.\\
\textbf{Løsning} \\
Public List search(String searchKey, String searchValue)\\
Der er blevet valgt at ”searchKey” argumentet, håndteres som en af to prædefinerede sætninger: ”case” eller ”citizen”.\\
Argumentet ”searchValue”, består af de attributter som bruges til at finde den ønskede data, samt autorisere personens ret til at se dem.\\
Ved brug af ”case” sætningen, aktiveres en SQL-Query, som leder efter en enkelt sag. Denne Query benytter attributter fra ”searchValue” til at udfylde metodens wildcards:\\
”caseid”(Erstatter de første tre wildcards) og ”departmentid”(Erstatter det fjerde wildcard).\\
Kontrollen af ”departmentid”, bruges til at autorisere et bosteds ansvarsområde i databasen. Hvis brugeren som søger på sagen, arbejder i den afdeling der behandler sagen, vil de kunne se data-outputtet der sendes med ”SearchCase” objektet.\\
Der bruges et ”caseid” til at identificere den specifikke sag der søges. Dette gøres ved hjælp af en Sub-Query, der sorteres efter ”datestamp” blandt de sager som har samme ”caseid”. Denne Sub-Query bruges til at hente de relevante data, fra den seneste udgave af en sag.
\begin{figure} [htb!]
\begin{lstlisting}
if(searchKey.equals("Case")) {
	try {
		connectToDB();
			if(dbConnection != null) {
				String[] search = searchValue.split("%");
				selectQuery = "SELECT c.caseid, c.casestatus, to_char(c.createddate, 'DD/MM-YYYY'),
				ci.citizenid, CONCAT(ci.firstname, ' ', ci.lastname) AS citizenName,
				to_char(cc.datestamp, 'DD/MM-YYYY') AS datestamp, cc.regardinginquiry,
				CONCAT(e.firstname, ' ', e.lastname) AS employeeName, e.employeeid ";
				
				fromQuery = "FROM \"case\" as c, citizen AS ci, (SELECT casecaseid, datestamp,
				regardinginquiry FROM case_contents WHERE casecaseid = ? ORDER BY datestamp DESC LIMIT 1)
				AS cc, (SELECT casecaseid, employeeemployeeid FROM case_employee
				WHERE casecaseid = ?) AS ce, employee AS e ";
				
				whereQuery = "WHERE ce.casecaseid = c.caseid AND e.employeeid = ce.employeeemployeeid 
				AND cc.casecaseid = c.caseid AND ci.citizenid = 
				c.citizenregardingcitizenid AND c.caseid = ? AND
				c.departmentdepartmentid = ? ORDER BY c.caseid DESC;";
				
				query = selectQuery + fromQuery + whereQuery;
				searchCaseStmt = dbConnection.prepareStatement(query);
				searchCaseStmt.setString(1, search[0]);
				searchCaseStmt.setString(2, search[0]);
				searchCaseStmt.setString(3, search[0]);
				searchCaseStmt.setInt(4, Integer.valueOf(search[1]));
				dbResultSet = searchCaseStmt.executeQuery();
				while(dbResultSet.next()) {
					sc.add(new SearchCase(dbResultSet.getInt(4), dbResultSet.getString(5), dbResultSet.
					getString(1), dbResultSet.getString(2), dbResultSet.getString(6),
					dbResultSet.getString(3), dbResultSet.getString(7), 
					dbResultSet.getInt(9), dbResultSet.getString(8)));
                    }
                }
\end{lstlisting}
\end{figure}
\\Ved brug af ”citizen” sætningen, aktiveres en SQL-Query, som søger data på baggrund af en borgers CPR-nummer navn, adresse, gadenummer og/eller postkode.\\
\begin{figure} [htb!]
\begin{lstlisting}
else if(searchKey.equals("Citizen")) {
            try {
                connectToDB();
                if(dbConnection != null) {
                    String[] search = searchValue.split("%");
                    List<String> searchVal = new ArrayList();
                    String searchQuery = "";
                    int columns = 0;
                    if(!search[0].equals("")) {
                        searchQuery += "\"CPR-nr\" LIKE ?";
                        columns++;
                        searchVal.add(search[0]);
                    } 
                    if(!search[1].equals("")) {
                        if(!searchQuery.equals("")) {
                            searchQuery += " AND ";
                        }
                         searchQuery += "CONCAT(ci.firstname, ' ', ci.lastname) LIKE ?";
                         columns++;
                         searchVal.add(search[1]);
                    } 
                    if(!search[2].equals("") || !search[3].equals("")) {
                        if(!searchQuery.equals("")) {
                            searchQuery += " AND ";
                        }
                        searchQuery += "(CONCAT(ci.streetname, ' ', ci.houseno) LIKE ?
                        AND CONCAT(ci.zipcodezipcode, ' ', z.cityname) LIKE ?)";
                        columns++;
                        searchVal.add(search[2]);
                        columns++;
                        searchVal.add(search[3]);
                    }
\end{lstlisting}
\end{figure}
\newpage
Denne Query benytter ”departmentid” til at autorisere søgningen, på samme måde som be-skrevet for første ”searchKey”-eksempel.\\
Attributterne der anvendes fra ”searchValue”, placeres i et String array og derefter samles i en liste. Elementerne fra denne liste, kaldes enekltvis mellem to wildcards (\%). Ved at tilføje disse wildcards på hver side, underrettes den pågældende SQL-Query om at der søges på dele af dataværdier. Når en String fra arrayet skal matches i databasen, vil alle elementer med samme værdi blive hentet, så længe at værdien er case sensitive.
(Se figur \ref{code:wild})\\
\begin{figure}
\begin{lstlisting}
searchCaseStmt = dbConnection.prepareStatement(query);
int index = 1;
for(int i = 1; i <= 3; i++) {
	for (int j = 0; j < columns; j++) {
		searchCaseStmt.setString(index, "%" + searchVal.get(j) + "%");
		index++;
	}
}
searchCaseStmt.setInt(index, Integer.valueOf(search[4]));
dbResultSet = searchCaseStmt.executeQuery();
while(dbResultSet.next()) {
	sc.add(new SearchCase(dbResultSet.getInt(4), dbResultSet.getString(5), 
	dbResultSet.getString(1), dbResultSet.getString(2), dbResultSet.getString(6), 
	dbResultSet.getString(3), dbResultSet.getString(7), dbResultSet.getInt(9), 
	dbResultSet.getString(8)));
s}
\end{lstlisting}
\caption{\% search}
\label{code:wild}
\end{figure}
Når søgefunktionen implementeres på denne måde, er det muligt at returnere en bred søg-ning. En List vil blive fyldt med alle elementer som er relevante i forhold til et givent søgeord.\\
Hvis der bliver søgt efter navnet ”Peter”, vil alle der hedder ”Peter” blive hentet. På grund af brugen af wildcards, vil navne hvor ”Peter” indgår i også blive hentet, f.eks:\\
”Anders Petersen”, ”Jens Peter”, ”Peter Halfdan”\\

\subsection{Unit test}
\subsubsection{saveCaseTest()}
Der er udført en test på saveCase() metoden  i klassen Department. Testens formål er at kunne modtage data fra Præsentationslag og lægge det ned til Persistenslaget som herefter bliver gemt i databasen.  \\
\begin{figure}[htb!]
\begin{lstlisting}
@Test
public void saveCaseTest() {
	Map<String, Map<String, String>> caseInfo = new HashMap<>();
	department.setDepartmentID(1);
	Map<String, String> conentsMap = new HashMap<>();
	conentsMap.put("regardinginquiry", "Henvendelse drejer sig om en Test");
	conentsMap.put("treatment", "paragrafTest2");
	conentsMap.put("carriage", "paragrafTest1");
	conentsMap.put("caseID", "-1");
	
	Map<String, String> citizenInfoRegarding1 = new HashMap<>();
	citizenInfoRegarding1.put("firstName", "Hans");
	citizenInfoRegarding1.put("lastName", "Frederiksen");
	citizenInfoRegarding1.put("streetName", "Snervej");
	citizenInfoRegarding1.put("cityName", "Taarup");
	citizenInfoRegarding1.put("zipCode", "5560");
	citizenInfoRegarding1.put("houseNo", "75");
	citizenInfoRegarding1.put("regardingCitizen", "true");
	citizenInfoRegarding1.put("requestingCitizen", "false");
	citizenInfoRegarding1.put("cpr", "887799-5544");
	citizenInfoRegarding1.put("floor", "");
	citizenInfoRegarding1.put("floorDirection", "");
	
	Map<String, String> citizenInfoRequesting1 = new HashMap<>();
	citizenInfoRequesting1.put("firstName", "Mads");
	citizenInfoRequesting1.put("lastName", "Simpson");
	citizenInfoRequesting1.put("streetName", "tillevej");
	citizenInfoRequesting1.put("cityName", "Odense");
	citizenInfoRequesting1.put("zipCode", "5000");
	citizenInfoRequesting1.put("houseNo", "22");
	citizenInfoRequesting1.put("regardingCitizen", "false");
	citizenInfoRequesting1.put("requestingCitizen", "true");
	citizenInfoRequesting1.put("cpr", "120596-7745");
	citizenInfoRequesting1.put("floor", "etage 2");
	citizenInfoRequesting1.put("floorDirection", "venstre");
	
	caseInfo.put("caseContents", conentsMap);
	caseInfo.put("cRegarding", citizenInfoRegarding1);
	caseInfo.put("cRequesting", citizenInfoRequesting1);
	
	assertEquals(true, department.saveCase(caseInfo));
}
\end{lstlisting}
\caption{Unit test kode for Savecase}
\label{kode:savetest}
\end{figure}\\
Der er blevet oprettet en metode saveCaseTest() (Se figur \ref{kode:savetest}), hvor der oprettes en caseInfo Map som skal kunne modtage en RegardingCitizen af typen Map og en RequestingCitizen af typen Map. Der bliver udover det også oprettet en caseContents af typen Map hvor alt data vedrørende sagsoplysninger lægges i. På figuren ses det at der er sat key og value værdier ind, og da det også skal være muligt at indsætte tomme pladser i f.eks. case\_contents tabellen i databasen, er caseContents ikke fyldt helt ud med alle værdier. Som det næste sættes en RegardingCitizen ind med oplysninger og en RequestingCitizen med oplysninger. Dette for at teste om borgen registreres i databasens Citizen tabel. Til sidst lægges alle 3 Maps ind i caseInfo Map hvor der efterfølgende bliver brugt en assertEquals() metode for at teste om det forventede resultat er sandt. \\
Efter at have kørt testen igennem returnerede den true hvilket betød at testen var succesfuldt. \\
\subsubsection{searchTestCase()}
Der er udført en test på search() metoden i Department, som skal vise at man får et søges resultat tilbage fra datalaget med det data der er blevet søgt på. I test metoden searchTestCase() er overvejelsen af at man kan søge på en sag med specifikt sagsnummer og få resultatet af søgning fra datalaget og hele vejen igennem op til brugergrænsefladen. \\
I figur \ref{kode:searchcase} fra interne bilag udføres testen searchTestCase() hvor der deklareres en variable caseNumber med værdien ”123” der skal være formålet med at søge en sag frem på et specifikt sagsnummer. Metoden searchHandler.search("Case", caseNumber + "\%1"), er den metode som sender nøglen der ”Case” og værdien ”123\%1”. Adskillelsen med \% fortæller at den første værdi er sagsnummeret og den anden er departmentID. Der oprettes et expected Map med de værdier der forventes at få. I assertThat(searchResultMap, Is.is(expectedMap)) metoden tjekker man om searchResultMap er det samme som det forventede.\\
Ved test af koden blev det forventede map printet ud og var det samme som funktionens map, hvilket gjorde at testen var succesfuldt. \\
Testen viser at der kan søges på en specifik sag og samtidig se hvilken borger der er tilknyttet sagen. Dette gør det lettere for at kunne finde ud af hvad sag en borger er tilknyttet.  \\
\subsubsection{searchTestCitizen()}
I det forrige  afsnit søgte metoden searchTestCase() på en specifik sag. Der har som en beslutning af gruppen også fortaget at implementere search() med at kunne søge på en borger. Testen searchTestCitizen() har til formål at kunne søge på en specifik borger og baseret på borgeren vise de sager den pågældende borger har. \\
I figur \ref{kode:searchcitizen} fra interne bilag anvendes searchTestCitizen() metoden hvor der bliver deklareret en variable citizen af typen String. Den får værdierne "123123-1231\%Poul Johanson\%kimvej\%", der bliver separeret med ”\%”. Derefter oprettes et Map som kaldes expectedMap der er data som er forventet at få tilbage. Derefter laves assertThat(searchResultMap, Is.is(expectedMap)) der er tester om searchResultMap er det samme som expectedMap. \\
Testen viser at det er den rigtige borger der kommer frem når man søger, hvilket gør at testen er succesfuldt 
\subsection{System test}
\subsubsection{Generelt}
\begin{itemize}
\item Systemet kan ikke startes ved at trykke på projektmappen og derefter run. Når dette gøres, starter den login systemet om, og man har mulighed for at indtaste sine logininformationer, der er i dette tilfælde blevet brugt ”admin” til både brugernavn og adgangskode, når dette er gjort og man trykker på loginknappen, bliver der i konsollen smidt en ”NullPointerException” men Location is required, og det er ikke muligt at komme vindere i systemet.
\item Kan ikke ud kommentere Main-metode i mmmi, hvis dette gøres, kan modulet ikke startes op. Det var forventet at samarbejdet mellem loginsystemet og mmmi modulet var sat op så mmmi ikke havde brug for en Main-metode.
\item I menuen i venstre side på UI’et er det kun Hjem, Opret sag og Find sag som kan trykkes på, resten af knapperne har ingen funktion.
\end{itemize}
\subsubsection{Login System}
\begin{itemize}
\item Når man har skrevet brugernavn eller adgangskode forkert 4 gange, der er i input til brugernavn og adgangskode blevet brugt henholdsvis ”admin” til brugernavn og ”ain” til adgangskode, skal man kontakte admin, men der er muligt at bypass ved at lukke programmet og starte det igen.
\end{itemize}
\subsubsection{Opstarts side}
\begin{itemize}
\item Når der er skrevet tekst i eventuelle noter og går videre til sig næste sag, rydder den ikke feltet eller skriver det som er på en anden sag. Der er dog skrevet og gemt i databasen.
\item Hvis man skriver i kommentarfeltet og reguardingquiry er ”Null” i databasen, der er blevet testet på sagen med sagsnummer ”123” som har noget stående i case\_contents tabellen. bliver der skrevet en SQL-fejl, men ”violates not-null constraint”.  Det forventet output var at det lavet en ny række i tabellen med et nyt timestamp og det eksisterne indhold med ændringer i den kolonne som hedder alternativenotes.  
\item Feltet hvor navnet på brugerne som er logget ind, er ikke stort nok, da der er en begrænsning og ikke kan vise ”Aleksander Henriksen”, men skriver ”Aleksander Henrik…”
\end{itemize}
\subsubsection{Sagsåbnings dokument}
\begin{itemize}
\item Henvendelse:
\begin{itemize}
\item Hvis man indtaster et postnummer som ikke eksistere i databasen, der er i testen indtastet ”1”, bliver der smidt en fejl med violates foreign key constraint, der mangler en håndtering.
\end{itemize}
\item Rettigheder og pligter:
\begin{itemize}
\item Er borgerne informeret om ret til bisidder og partsrepræsentant, der mangler mulighed for at svar ja eller nej eller så mangler der at gøres så det er muligt at fjerne markeringen af dette ved fejl udfyldning.
\end{itemize}
\item Vægemål og repæsentant:
\begin{itemize}
\item Teksfelt under f.eks. vægemål og repreæsentant hvis der vælges samværgemål aktiveres teksfeltet, men hvis man vælger f.eks. værgemål så er indholdet stadig i feltet hvis der trykkes på gem. 
\end{itemize}
\item Ydelser og paragraffer:
\begin{itemize}
\item Under de forskellige punkter er det ikke muligt at fjerne sin markering helt i tilfælde af fejltryk eller borgerne ikke skal have denne ydelse, det er dog lavet på nogle få stykker.
\item Hvis man marker flere ydelser under de forskellige sektioner forekommer problemet at der ikke gemmes alle værdiger fra de afkrydsede radiobuttons, da der er for mange der er blevet afkrydset. 
\end{itemize}
\item Når man gemmer:
\begin{itemize}
\item Der skal trykkes på 2 knapper for at gemme et dokument, og når man trykker på den sidste gem knap, bliver dataene som man vil gemme lagt som Null eller tomt i databasen.
\item Når man indtaster information omkring en borger, og man trykker på gem 2 gange, bliver der anden gang man trykker gem smidt en ”NullPointerException”, og der bliver kun oprettet en sag selvom det kunne være 2 forskellige sager på samme borger, men der bliver oprettet 2 borger i Citizen tabellen, med de samme værdier.
\end{itemize}
\end{itemize}
\subsubsection{Udredning dokument}
\begin{itemize}
\item Det er ikke muligt at lavet et tab og komme videre til næste tekst felt.\end{itemize}
\subsubsection{Find sag}
\begin{itemize}
\item Når der søges på sagsnummer ”123” som er et case id i databasen, bliver der skrevet en fejl på at integer = character varing, så der er ikke blevet håndteret at caseid i database er blevet serial i stedet for varchar, derved ingen output i UI’et.
\item Når der søges efter ”peter” eller ”peter test” under feltet Borgers navn og trykker på knappen Søg, bliver der ikke fundet nogen sager og vist i tableviewet. Det forventet output var to sager på Peter men sagsnumrene 37 og 38.
\item Det samme sker når man søger på ”snervej” under feltet Borgers adresse og trykker på kanppen Søg. Det forventet output var 5 sager med sagsnumrene 46-50.
\item Skrives der ”800” eller ”800 Høje Taastrup” i feltet postnummer og by, vises der ingenting i tableviewet, men et komma adskilles af postnummer og by, gives der heller ikke nogen søgning. Det forventet output var en sag på Jesper Jensen men sagsnummer 28 og to sager på Peter Test med sagsnumrene 37 og 38.
\item Ved indtastning i CPR-nummer, med CPR-nummeret ”120366-1235” eller ” 1203661235” eller ” 120366 1235”, gives der heller noget resultat, der var ved denne søgning forventet at få vist en sag med sagsnummer 1, en oprettelse dato: 21-05-2019, borgers navn: Jens Hansen og ingen ting i resten af felterne, da der ikke er noget indtastet data dertil.
\end{itemize}

