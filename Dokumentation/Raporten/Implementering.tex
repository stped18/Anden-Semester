\subsection{Implementering}
Målet med første iteration var at skabe et domænelag, der havde de funktionaliteter, som var nødvendige for at systemet kunne fungere. Dette blev gjort så gruppen havde en bedre forståelse for hvordan systemet skulle se ud. 
Gennem første iteration blev der lavet to klasse (case og department) der implementerede begge brugsmønstre. ”Case” håndterede hvilke data der skulle gemmes i en sag, mens ”department” holdt styr på hvilke specifikke sager der var og hvordan de skulle tilgås.  Nedenfor ses de forskellige attributter i ”Case” klassen. (Se figur \ref{kode:caseatt})
\begin{figure}[h]
\begin{lstlisting}
public class Case {
    private static int socialCount = 1;
    private static int handicapCount = 1;

    private final String caseNumber;
    private String caseStatus;

    private final String date;
    private final Citizen regardingCitizen;
    private final List<Citizen> requestingCitizen = new ArrayList<>();
    private final Map<String, String> information = new HashMap<>();
    private final int departmentID;
\end{lstlisting}
\caption{Case klassens attributter}
\label{kode:caseatt}
\end{figure}\\
Der bruges et map til at holde den data der ændre og skrives ind.\\ 
Klassen indeholder en status på om sagen er lavet, i gang eller færdig.\\ 
Der kan kun være én person sagen omhandler, så der er en instans af klassen. Der kan være flere der henvender sig angående den samme person. Dette blev håndteret ved at lave list over disse personer. 
\begin{figure}
\begin{lstlisting}
public List<Case> search(String searchWord) {
	List<Case> listCases = new ArrayList<>();
	
	for (Case searchCase : caseMap.values()) {
		if (searchCase.getRegardingCitizen().getCprNumber().equalsIgnoreCase(searchWord)
				|| searchCase.getCaseNumber().equalsIgnoreCase(String.valueOf(searchWord))
				|| searchCase.getRegardingCitizen().getName().equalsIgnoreCase(searchWord)) {
            listCases.add(searchCase);
		}
	}
	return listCases;
}
\end{lstlisting}
\caption{Department search, finder alle sager der har søgeordet i}
\label{kode:search}
\end{figure}
Den data der kan vises i præsentationslaget kaldes igennem facaden "Department". Det var også her at alle de forskellige sager blev gemt i denne iteration. Der bliver genereret et sagsnummer hvor det er muligt at se hvilken afdeling sagen tilhører. \\ 
I figur \ref{kode:search} kan det ses hvordan klassen ”Department” finder en sag. Den skal bruge en ”string” som kunne være CPR-nummer eller navn for personen den omhandler eller  sagsnummeret. \\
Som det kan ses i koden er bliver der ikke returneret en sag, men en liste af sager.\\
Dette valgte gruppen at gøre fordi der kan søges på et navn, som ikke er unikt og der kan derfor findes mere end en persons sager. Samtidig er det også muligt at en person har mere end en sag, så selv hvis der kun er én person med et givent navn eller der søges på et CPR-nr kan der stadig returneres mere end én sag. Dette nåede ikke at blive implementeret i denne iteration. 

