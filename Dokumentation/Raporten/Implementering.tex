\section{Implementering}



\subsection{Find sag metoden}
Under Department.java har vi en metode 
\subsubsection{Problem: }
En aktør skal kunne finde en sag samt få denne sag vist. Sagen skal findes ved hjælp at enten sagsnummer, cpr-nummer eller navn. 
\subsubsection{Løsning: }
For at finde en sag skal der første laves en liste over sagerne der er inde for den gældende afdeling har. 
Der bliver lavet en metode search der søger gennem alle de sager afdelingen har,
Herefter skal der findes en match på om sagen har enten matchende sagsnummer, cpr-nummer eller navn. 
\begin{figure}[h]
\begin{lstlisting}

    public List<Case> search(String searchWord) {

        List<Case> listCases = new ArrayList<>();

        for (Case searchCase : caseMap.values()) {
            if (searchCase.getRegardingCitizen().getCprNumber().equalsIgnoreCase(searchWord)
                    || searchCase.getCaseNumber().equalsIgnoreCase(String.valueOf(searchWord))
                    || searchCase.getRegardingCitizen().getName().equalsIgnoreCase(searchWord)) {
                listCases.add(searchCase);
            }
        }

        return listCases;
    }


\end{lstlisting}
\caption{kode : seartch}
\label{kode:Search}
\end{figure}


\subsubsection{Evaluering:}
Metoden virker når der kun er kendskab til et map men skal laves om når der skal søges i databasen efter den gældende sag.