\chapter{Indledning}
EG Team Online har været med til at udarbejde en case (ekstern bilag \ref{eks:Case}), hvori de ønsker en belysning af 3 moduler i deres system, Sensum. Gruppen har af de 3 moduler valgt at lægge vægt på modulet sagsudredning. Udviklingen af modulet vil følge arbejdsprocessen UP (Se metode afsnit \ref{sec:samlede metoder}). Dette modul er bygget til sagsbehandlere hos kommunerne og er opbygget efter VUM’s retningslinjer. (Anon., 2019) Denne metode bliver brugt til at visitere voksne med et handicap eller misbrug til ydelser, der kan hjælpe dem i deres hverdag.
\section{Resume af case} (ref til case)
Gruppen har fået udleveret en case som omhandler systemerne Sensum og Sensum Bosted som er udviklet af EG Team Online. Sensum Bosted og Sensum er systemer der understøtter det samlede behov for dokumentationsstyring i den offentlige og private social- og sundhedssektor.\\
Systemet har hidtil været opdelt i to forskellige systemer det ene er Sensum Bosted og det andet er Sensum, begge dele er modulopbygget plug-and-play, som gør det muligt at tilpasse systemet til kundens behov. Sensum Bosted er det system som har fokus på at digitalisere den daglige arbejdsproces på bostederne, og understøtter alle former for dokumentationsbehov på socialområdet. 
Sensum er et udredningssystem der skaber et overblik over sagsbehandlingen. 
Systemet er et DHUV-system 
”Digitalisering af Handicap og Udsatte Voksne”, som er en proces beskrevet under VUM  (Anon., 2019)
\\
Virksomheden ønsker en belysning af tre moduler: planlægning, sagsudredning og dagbog. \\
Planlægningsmodulet er et værktøj hvor et botilbud kan tilrettelægge en borgers hverdag igennem en illustration med piktogrammer, som beskriver de aktiviteter borgeren skal nå den pågældende dag.
Sagsudredningsmodulet er den del af systemet hvori sagsbehandlere på en kommune udreder borgerne jfr. VUM 
(Anon., 2019), %kilde sæt ind rigtig
og tildeler ydelser som f.eks. behandling på et misbrugscenter eller botilbud for handicappet.\\
Dagbogsmodulet bruges til journalføring på borgerne. Der bliver både skrevet i journal under udredningen og på botilbuddet som daglig notatføring. Dele af dagbogen bliver automatisk genereret af system, det er f.eks. når borgeren er tildelt en ydelse eller når der bliver udleveret medicin.
\newpage
\section{Problemformulering}
Gruppen har diskuteret afgrænsning i forhold til de specifikke moduler. I forhold til persondatalovgivningen har virksomheden beskrevet en dataafgrænsning i casen hvor de beskriver hvem der må kunne interagere med hvad i systemet. På baggrund af de strengere regler om persondata, blev det nødvendigt at opdatere systemet.

\textbf{\\Gruppen kom frem til følgende hovedspørgsmål.}\\ \\
\noindent\fbox{%
    \parbox{\textwidth}{%
Hvordan kan et program håndtere flere roller uden at de interagerer med hinandens data samt hvilke sikkerhedsforanstaltninger ligger til grund for at beskytte data?
    }%
}
 \textbf{\\Delspørgsmål\\}\\
På baggrund af den centrale problemstilling er der blevet udformet følgende underspørgsmål:\\ 
\begin{itemize}
\item Hvad definerer roller?
\item Hvilket ansvarsområder ligger til grund for rollerne?
\item Hvilken persistent data bearbejdes?
\item Hvordan håndterer systemet flere roller?
\item Hvordan håndteres roller med flere ansvarsområder?
\item Hvilke sikkerhedsforanstaltninger er der i forhold til data afgrænsningen?
\item Hvordan sikres det at aktøren kun har adgang til det aktøren har brug for? 
\end{itemize}
Gruppen analyserede den udleverede projektcase. På baggrund af analysen dannede gruppen et overblik over problemstillingen. Analysen gav en forståelse for de forskellige aktørers roller og ansvarsområder. EG Team Online ønsker en belysning af 3 moduler, hvor gruppen valgte dataafgrænsningen i forhold til sikkerhedsforanstaltninger, der adskiller de forskellige roller.
\newpage
\section{Formål med projekt}
Formålet med projektet er at udvikle et udkast til et eller flere af de moduler som er beskrevet i den udleveret case. 
Her har gruppen så valgt at ligge vægt på modulet sagsudredning hvor fokuspunktet er at lavet et system med en dataafgrænsning.\\
Gennem processen med udviklingen af modulet skal der blive arbejdet med softwareudviklingsmodelen UP. Der er i løbet af projektet blevet udarbejdet arkitektoniske og analytiskes modeller, samt desginmodeller til at understøtte tanken med systemet. Disse diagrammer er blevet udarbejdet på baggrund af den kravspecifikation som gruppen er nået frem til. 
\section{Målene med projektet}
Der ligges primært vægt på teorien fra Organisation og Grundlæggende Software Engineering. Projektet går hovedsageligt ud på at analyser ansvarsområdet, med henblik på at identificere strukturer og organisatoriske processer. 
For at udvikle det modul som gruppen er kommet frem til er der blevet arbejdet med UP. 
Der er i forlængelse af UP blevet brugt dele af Scrum.
(Se afsnit \ref{sec:samlede metoder})%find det rigtige afsnit og lav ref
\\
I udviklingen af modulet, sagsudredning, skal der indrages elementer fra Videregående Objektorienteret Programmering(VOP) og Database, samt arbejdes ud fra en 3-lags arkitektur, hvilket gør at systemet består af et præsentations-, domæne- og persistenslag. 
For at udarbejde præsentations- og domænelaget skal der bruges dele af teorien fra VOP. 
For at udarbejde persistens laget skal der bruges dele af teorien fra både VOP og Database, hvor Database ligger vægt på database design.


