Giver indledningen et overblik over projektet og baggrunden for det?\\
Giver indledningen et resume af den udleverede case?\\
Indeholder indledningen problemformuleringen, jfr. inceptiondokumentet?\\
Beskriver indledningen formålet med projektet?\\ Er formålet i overensstemmelse med hensigten med 2. semester? \\
Beskriver indledningen målene med projektet? Udtrykker målene specifikke, målbare resultater, jfr. inceptionsdokumentet? \\
Er målene i overensstemmelse med de overordnede mål for 2. semester som udtrykt i studieordningens kap. 9 og de mere specifikke mål for projektet, som udtrykt i fagbeskrivelsen for SI2-PRO?

\section{Problemformulering}


Salget af KMD førte til etableringen af det kommunale IT-fællesskab KOMBIT\footnote{\url{ https://www.kombit.dk/KOMBIThistorie}}, som arbejder på at sikre de bedste og billigste IT-løsninger set ud fra kommunernes behov. De kommunale behov blev tidligere varetaget af KMD, men som resultat af deres monopolbrud, valgte KOMBIT at udbyde disse behov til forskellige IT-leverandører . Dette skabte muligheden for at EG Team Online kunne træde ind på det kommunale IT-marked.
Allerede i 2005 præsenterede EG Team Online, dengang kendt som Team Online, deres første store system inden for velfærdsområdet, Sensum Bosted, dengang kendt som Bosted System. Dette system var revolutionerende inden for lokale bosteder og institutioner som arbejder med patientbehandling. Ved at digitalisere journaler og andre former for papirarbejde, blev socialarbejdernes arbejdsbyrde mindsket og deres effektivitet øget. I 2007 modtog Team Online ansvaret for it-løsninger hos Fyns Amt bo- og aktivitetsstedet Lindebjerg. Efterfølgende har virksomheden udviklet sig til at være en leverandør inden for innovativ teknologi med speciale i løsninger, der understøtter det samlede behov i den offentlige og private social- og sundhedssektor.\\ \\
I 2018 blev GDPR indført af EU-kommissionen, hvilket gjorde at den danske regering tilpassede den danske datalovgivning til GDPR. Denne ratificering betød at alle IT-selskaber skulle følge nye regler angående håndtering af persondata. Derfor er EG Team Online ved at tilpasse deres produkt.\\
\\
Gruppen har fået udleveret en case som omhandler et system udviklet af EG Team Online. Virksomheden ønsker en belysning af tre moduler: planlægning, sagsudredning og dagbogen. Gruppen har diskuteret afgrænsning i forhold til de specifikke moduler, hvor der var en interesse i hvordan flere roller i systemet interagerer med hinanden. I forhold til persondatalovgivningen har virksomheden beskrevet en dataafgrænsning i casen hvor de beskriver hvem der må kunne interagere med hvad i systemet. På baggrund af de strengere regler om persondata, blev det nødvendigt at opdatere systemet.
 

\textbf{\\Gruppen kom frem til følgende hovedspørgsmål.}\\ \\
\noindent\fbox{%
    \parbox{\textwidth}{%
Hvordan kan et program håndtere flere roller uden at de interagerer med hinandens data samt hvilke sikkerhedsforanstaltninger ligger til grund for at beskytte data?
    }%
}
 \textbf{\\DeleSpørgsmål\\}\\
På baggrund af den centrale problemstilling er der blevet udformet følgende underspørgsmål:\\ 
\begin{itemize}
\item Hvad definerer roller?
\item Hvilket ansvarsområder ligger til grund for rollerne?
\item Hvilken persistent data bearbejdes?
\item Hvordan håndterer systemet flere roller?
\item Hvordan håndteres roller med flere ansvarsområder?
\item Hvilke sikkerhedsforanstaltninger er der i forhold til data afgrænsningen?
\item Hvordan sikres det at aktøren kun har adgang til det aktøren har brug for? 
\end{itemize}
Gruppen lavede en analyse af den udleverede case. På baggrund af analysen dannede gruppen et overblik over problemstillingen. Analysen belyser de forskellige aktørers roller og ansvarsområder. EG-teamet ønsker en belysning af 3 moduler, hvor gruppen valgte dataafgrænsningen i forhold til sikkerhedsforanstaltninger, der adskiller de forskellige roller.




\section{Metoder}
Er metoden i det samlede projekt beskrevet?\\
Er det beskrevet hvordan UP og Scrum kombineres i projektet, samt hvilke fordele og ulemper der er ved det?\\

\subsection{Indledning til metoder}
Giver indledningen en introduktion til  afsnittet?\\
\subsection{Metoder}
Er metoden i det samlede projekt beskrevet?\\
Er det beskrevet hvordan UP og Scrum kombineres i projektet, samt hvilke fordele og ulemper der er ved det?\\
\subsection{Planlægning}
Er planlægningen af elaborationsfasen og de enkelte iterationer beskrevet.\\
Er backlogs beskrevet?\\
Er rollefordelingen i projektgruppen beskrevet?\\
Er ceremonierne beskrevet?\\
Er scrum-buts beskrevet?\\
Bygger planen på prioriteringen af kravene efter inceptionsfasen.\\
