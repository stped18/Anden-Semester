\chapter{Indledning}
Gruppen har fået udleveret en case som omhandler systemerne Sensum Bosted og Sensum som er udviklet af EG Team Online. Sensum Bosted og Sensum er systemer der understøtter det samlede behov i den offentlige og private social- og sundhedssektor.\\
Systemet har hidtil været opdelt i to forskellige systemer det ene er Sensum Bosted og det andet er Sensum, begge dele er modulopbygget plug-and-play, som gør det muligt at tilpasse systemet til kundens behov. Sensum Bosted er det system som har fokus på at digitalisere den daglige arbejdsproces på bostederne, og understøtter alle former for dokumentationsbehov på socialområdet. Sensum er et sags- og udredningssystem der skaber et overblik over sagsbehandlingen, faciliterer og kommunens arbejdsprocesser. Systemet er et DHUV-system 
\footnote{\url{https://socialstyrelsen.dk/tvaergaende-omrader/sagsbehandling/vum-og-dhuv}} 
 ”Digitalisering af Handicap og Udsatte Voksne”, som en proces beskrevet under VUM ”Voksen Udrednings Modellen”.\\
Virksomheden ønsker en belysning af tre moduler: planlægning, sagsudredning og dagbogen.\\
Planlægningsmodulet er et værktøj hvor et botilbud kan tilrettelægge en borgers hverdag igennem en illustration med piktogrammer, som beskriver de aktiviteter borgeren skal nå den givende dag.\\
Sagsudredningsmodulet er den del af systemet hvor i sagsbehandlerne på kommunen udreder borgerne jfr. VUM-metoden 
\footnote{\url{https://socialstyrelsen.dk/tvaergaende-omrader/sagsbehandling/vum-og-dhuv}}
 , og tildeler ydelser som f.eks. behandling på et misbrugscenter eller botilbud for handicappet.\\
Dagbogsmodulet bruges til journalføring på borgerne. Der bliver både skrevet i journal af sagsbehandlerne i løbet af udredningen og på botilbuddet som daglig notatføring. Dele af dagbogen bliver automatisk genereret af system, det er f.eks. når borgeren er tildelt en ydelse eller når der bliver udleveret medicin.\\

\section{Problemformulering}

Gruppen har diskuteret afgrænsning i forhold til de specifikke moduler, hvor der var en interesse i hvordan flere roller i systemet interagerer med hinanden. I forhold til persondatalovgivningen har virksomheden beskrevet en dataafgrænsning i casen hvor de beskriver hvem der må kunne interagere med hvad i systemet. På baggrund af de strengere regler om persondata, blev det nødvendigt at opdatere systemet.

\textbf{\\Gruppen kom frem til følgende hovedspørgsmål.}\\ \\
\noindent\fbox{%
    \parbox{\textwidth}{%
Hvordan kan et program håndtere flere roller uden at de interagerer med hinandens data samt hvilke sikkerhedsforanstaltninger ligger til grund for at beskytte data?
    }%
}
 \textbf{\\DeleSpørgsmål\\}\\
På baggrund af den centrale problemstilling er der blevet udformet følgende underspørgsmål:\\ 
\begin{itemize}
\item Hvad definerer roller?
\item Hvilket ansvarsområder ligger til grund for rollerne?
\item Hvilken persistent data bearbejdes?
\item Hvordan håndterer systemet flere roller?
\item Hvordan håndteres roller med flere ansvarsområder?
\item Hvilke sikkerhedsforanstaltninger er der i forhold til data afgrænsningen?
\item Hvordan sikres det at aktøren kun har adgang til det aktøren har brug for? 
\end{itemize}
Gruppen lavede en analyse af den udleverede case. På baggrund af analysen dannede gruppen et overblik over problemstillingen. Analysen belyser de forskellige aktørers roller og ansvarsområder. EG-teamet ønsker en belysning af 3 moduler, hvor gruppen valgte dataafgrænsningen i forhold til sikkerhedsforanstaltninger, der adskiller de forskellige roller.





