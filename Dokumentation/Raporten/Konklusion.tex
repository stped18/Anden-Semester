\chapter{Konklusion}
For at sikre at programmet kan håndtere flere roller er det nødvendigt at vide, hvad en rolle er. 
Dette er blevet lavet i gruppens program, så der er tre forskellige roller en bruger kan have mens de benytter systemet. 
Rollerne er sekretær, sagsbehandler og afdelingsleder. \\
Sekretæren har mindst ansvar, mens afdelingslederen har mest. 
Et område som en sekretær ikke har adgang til, men som de andre har, er at behandle en sag og skrive information ind i sagen. 
Systemet håndterer roller ved at tildele dem, når en person logger på. 
Et login har en bruger med en specifik rolle, der så fortæller systemet hvad brugeren må og ikke må. 
De fleste felter i databasen er i et string format og nogle af disse omhandler personfølsomme data. 
Der benyttes prepared statements til at skrive alt data til databasen, hvilket sikrer at en bruger ikke kan skrive SQL direkte til databasen. 
Funktioner til tildeling af sager eller afslutningen af sager har gruppen ikke nået at tilføje i systemet og så mangler der en en færdig implementering af rettighederne. 
Dette gør, at det lige nu er ligegyldigt hvad for en rolle brugeren har, men grundlaget er blevet lagt for at de bliver implementeret i fremtidige iterationer. 
Der er blevet implementeret begrænsninger så brugere ikke kan søge igennem andre afdelingers sager. 
Dermed er der fundet frem til en løsning på casens problem samt gruppens problemstilling. 
Modulet har en række ”TODO”’s i koden, hvilket gør det muligt at se i hvilken retning gruppen havde tænkt sig at arbejde i fremtidige iterationer.\\
