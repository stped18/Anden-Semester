\chapter{Læsevejledning}
Denne rapport er skrevet ud fra antagelsen om at læseren har en grundlæggende forståelse for programmering og software engineering, som svarer til gruppens eget vidensniveau. 
Det er meningen, at rapporten skal kunne læses igennem fra indledning til perspektivering. 
Indledning og konklusion kan i øvrigt læses i sammenhæng for at få en klar forståelse for projektets mål og de opnåede resultater. 
Ved læsning af første og anden iteration er de begge delt op i lignende afsnit. 
Afsnittene kan læses for en iteration ad gangen eller det ene afsnit fra første iteration og det samme afsnit fra anden iteration. 
Anden del af rapporten er om, 
hvad gruppen har lært og hvordan gruppen har arbejdet og hvordan samarbejdet har fungeret. For at se kildekoden anbefales det at se Javadoc i projektet. 
Der er også en brugervejledning der kort fortæller hvordan programmet skal bruges. 
Referencer til gruppens eget arbejde, som ikke kan ses på den pågældende side, 
kan findes i det interne bilag, 
mens øvrige bilag vil være i eksterne bilag. Dette er gældende for alle afsnit af rapporten.\\
