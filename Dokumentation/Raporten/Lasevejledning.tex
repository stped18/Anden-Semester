\chapter{Læsevejledning}
Denne rapport er skrevet med tanken om at læseren har en basisk forståelse for programmering, som svare til anden semester på universitetet. Det er meningen at rapporten skal kunne læses igennem fra indledning til perspektivering. Det er også meningen at for at få en klar forståelse for hvad gruppens mål var og hvad gruppen kom frem til at indledningen og konklusionen giver et indblik på hvad der står I rapporten og hvad der er kommet frem til. \\
Ved læsning af første og anden iteration er de begge delt op I ligne afsnit. Afsnittene kan læses for en iteration ad gangen eller det ene afsnit fra første iteration og det samme afsnit fra anden iteration. \\
Anden del af rapporten er hvad gruppen har lært og hvordan gruppen har arbejdet, både sammen og hvordan arbejdes gangen har været. \\
For at se kildekoden anbefaledes det at se javadoc I projektet. Der er også en brugervejledning der kort fortæller hvordan programmet er tiltænkt at det skulle bruges. For at se logbogen for projektet skal der bruges internet, med et link I afsnitet til det. \\
Hvis der er refereret til noget gruppen har lavet og det ikke kan ses på siden står det formentlig I interne bilag, mens udefra kommende bilag vil være I eksterne bilag. Dette er gældende I alle afsnit af rapporten. \\
