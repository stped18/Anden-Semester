\chapter{Metoder}
Metoder er en vigtig del af et projekt, som udformer væsentlige resultater og holder styr på fremgangen. Der vil under metodeafsnittet beskrives de metoder som har været anvendt under hele projektet, som f.eks. UP, kanban og par produktion. (Afsnit 7.1) \\
Unified Process (forkortet UP) er en iterativ, inkrementel og brugsmønster-drevet software udviklingsproces. UP bliver brugt med en kombination af den agile metode Scrum der er en trinvis og iterativ tilgang til udarbejdelse af et produkt. Gruppens anvendelse og kombination af disse beskrives. (Afsnit 7.2) \\
Til sidst bliver der uddybet beskrivelserne af Scrum’s forskellige egenskaber som f.eks. Scrum backlog, ceremonier m.m. (Afsnit 7.3) \\

\section{Samlede metoder}
Under starten af projektet blev der holdt et møde omkring hvordan og hvilket metoder der kunne og skulle anvendes. Under dette møde blev der forklaret hvilke metoder hvert enkelt gruppemedlem anvendte i deres projekt. Til det blev der diskuteret hvilke metoder der med fordel kunne anvendes i nuværende projekt, hvor der blev udvalgt de mest væsentlige og effektive metoder. Ydermere blev der gjort krav til at anvende UP og Scrum som læner sig op ad den agile metodologi. \\
\textbf{ Unified process (UP)} har været anvendt igennem hele projektet der er blanding af agile og planlægnings metode. Den består af forskellige faser, dvs. forberedelsesfase (inception), etableringsfase (elaboration), konstruktionsfase (construction) og overdragelsesfase (transition). Projektetrammen begrænser projektet i det omfang at forberedelsesfasen og etableringsfasen er de faser der skal fokuseres på. Dette fokus er bestemt for at tilrettelægge et projekt uden at fokus går på programmering selv.\\ 
\emph{Inceptionsfasen} er en kort fase der har til formål at give et overblik over de indhentede krav. Hovedtrækkene i fasen handler om at:\\
\begin{itemize}
\item Forstå hvad der skal bygges
\item Identificere de væsentlige funktioner i systemet, og beskrive dem ved at bruge brugsmønstre
\item Identificere projektets plan og kritiske risici
\item Udvælge udviklingsværktøjer til selve udviklingsprocessen
\end{itemize}
\emph{Elaborationsfasen} handler om at analysere problemdomænet og tilegne sig en endelig forståelse af hele systemet. Hovedtrækkene handler om at: \\
\begin{itemize}
\item Designe brugsmønstrene
\item Producere en arkitektonisk baseline for systemet som skal bygges
\end{itemize}
\textbf{Kanban} er en mindre formel metode end Scrum der bliver brugt til at holde styr på hvilke opgaver der skal laves og hvem der arbejder på opgaverne gennem hele projektet. Formålet med at anvende kanban i projektet har været at holde styr på de opgaver der skulle bearbejdes i løbet hver iteration. Den har fungeret effektivt i forhold til opdeling af opgaverne, hvor hvert par kunne blive tildelt en opgave. Til fordel for kanban blev værktøjet GitKranken Glo brugt som et virtuelt whiteboard for at holde styr på de forskellige opgaver og milepæle. \\
\\
\textbf{Par produktion} er en agil metode, beskrivelsen af definitionen kan findes i bilag … under metodeafsnittet. \\
Metoden er blevet anvendt under hele projektet, som har haft den fordel at udarbejdelsen af opgaverne blev mere effektive og produktive. Samtidigt blev der holdt styr på de opgaver der skulle arbejdes på hver uge. 
\\ \\
\textbf{Scrum} er en agil metode, beskrivelsen af Scrum kan findes i bilag … under metodeafsnittet.
I projektgruppen er der blevet besluttet at bruge dele af Scrum, der består af roller, herunder et team som består af gruppemedlemmerne. Ceremonier herunder daglige Scrum møder. Artefakter herunder produkt backlog. Det er ikke blevet fundet nødvendigt at anvende hele Scrum frameworket er ikke nok mennesker til at kunne bruge Scrum i det omfang at kunne benytte hele Scrum. Ydermere har vi ikke den nødvendige viden til at kunne benytte det, f.eks. har vi ikke en Scrum master eller product owner. \\

\subsection{Kombination af Scrum og UP}
Igennem hele projektforløbet er der blevet anvendt Unified Process. \\
Det betyder at der arbejdes i faser, hvor gruppen ender med at være igennem Inception- og elaborationsfasen. \\
Efter introduktionen til Scrum, er der i gruppen blevet valgt at kombinere noget af Scrum metoden med resten af metoderne. \\
Der anvendes Scrumbot, hvilket betyder at der er lavet en liste over spints og de iterationer som der udarbejdes igennem sprintene.\\
\subsection{Planlægning}
Er planlægningen af elaborationsfasen og de enkelte iterationer beskrevet.\\
Er backlogs beskrevet?\\
Er rollefordelingen i projektgruppen beskrevet?\\
Er ceremonierne beskrevet?\\
Er scrum-buts beskrevet?\\
Bygger planen på prioriteringen af kravene efter inceptionsfasen.\\