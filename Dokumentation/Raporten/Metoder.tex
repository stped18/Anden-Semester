\chapter{Metoder} \label{cap:metoder}
Metoder er en vigtig del af et projekt, som udformer væsentlige resultater og holder styr på fremgangen. Der vil under metodeafsnittet beskrives de metoder som har været anvendt under hele projektet, som f.eks. UP, kanban og par produktion. (Se Afsnit \ref{sec:samlede metoder}) \\
Unified Process (forkortet UP) er en iterativ, inkrementel og brugsmønster-drevet software udviklingsproces. UP bliver brugt med en kombination af den agile metode Scrum der er en trinvis og iterativ tilgang til udarbejdelse af et produkt. Gruppens anvendelse og kombination af disse beskrives. (Se Afsnit \ref{sec:scrum og up}) \\
Til sidst bliver der uddybet planlægningen af elaborationsfasen, m.m. (Se Afsnit \ref{plan})

\section{Samlede metoder} \label{sec:samlede metoder}
Under starten af projektet blev der holdt et møde omkring hvordan og hvilket metoder der kunne og skulle anvendes. Under dette møde blev der forklaret hvilke metoder hvert enkelt gruppemedlem anvendte i deres projekt. Til det blev der diskuteret hvilke metoder der med fordel kunne anvendes i nuværende projekt, hvor der blev udvalgt de mest væsentlige og effektive metoder. Ydermere blev der gjort krav til at anvende UP og Scrum som læner sig op ad den agile metodologi.  \\ \\
\textbf{ Unified process (UP)} har været anvendt igennem hele projektet der er blanding af agile og planlægnings metode. Den består af forskellige faser, dvs. forberedelsesfase (inception), etableringsfase (elaboration), konstruktionsfase (construction) og overdragelsesfase (transition). Projektetrammen begrænser projektet i det omfang at inceptionsfasen og elaborationsfasen er de faser der skal fokuseres på. Dette fokus er bestemt for at tilrettelægge et projekt uden at fokus går på programmering selv. \\ 
\emph{Inceptionsfasen} er en kort fase der har til formål at give et overblik over de indhentede krav. Hovedtrækkene i fasen handler om at:
\begin{itemize}
\item Forstå hvad der skal bygges
\item Identificere de væsentlige funktioner i systemet, og beskrive dem ved at bruge brugsmønstre
\item Identificere projektets plan og kritiske risici
\item Udvælge udviklingsværktøjer til selve udviklingsprocessen
\end{itemize}
\emph{Elaborationsfasen} handler om at analysere problemdomænet og tilegne sig en endelig forståelse af hele systemet. Hovedtrækkene handler om at: 
\begin{itemize}
\item Designe brugsmønstrene
\item Producere en arkitektonisk baseline for systemet som skal bygges
\end{itemize}
\textbf{Kanban} er en mindre formel metode end Scrum der bliver brugt til at holde styr på hvilke opgaver der skal laves og hvem der arbejder på opgaverne gennem hele projektet. Formålet med at anvende kanban i projektet har været at holde styr på de opgaver der skulle bearbejdes i løbet hver iteration. Den har fungeret effektivt i forhold til opdeling af opgaverne, hvor hvert par kunne blive tildelt en opgave. Til fordel for kanban blev værktøjet GitKranken Glo brugt som et virtuelt whiteboard for at holde styr på de forskellige opgaver og milepæle. \\
\\
\textbf{Par produktion} er en agil metode, beskrivelsen af definitionen kan findes i bilag … under metodeafsnittet. \\
Metoden er blevet anvendt under hele projektet, som har haft den fordel at udarbejdelsen af opgaverne blev mere effektive og produktive. Samtidigt blev der holdt styr på de opgaver der skulle arbejdes på hver uge. 
\\ 
\textbf{Scrum} er en agil metode, beskrivelsen af Scrum kan findes i bilag … under metodeafsnittet.
I projektgruppen er der blevet besluttet at bruge dele af Scrum, der består af roller, herunder et team som består af gruppemedlemmerne. Ceremonier herunder daglige Scrum møder. Artefakter herunder produkt backlog. Det er ikke blevet fundet nødvendigt at anvende hele Scrum frameworket er ikke nok mennesker til at kunne bruge Scrum i det omfang at kunne benytte hele Scrum. Ydermere har vi ikke den nødvendige viden til at kunne benytte det, f.eks. har gruppen ikke en Scrum master eller product owner. \\
\textbf{Anvendelse af UP og par produktion i projektet}
UP er den metode som har været anvendt med henblik på at lære hvordan metoden fungere både teoretisk og i praksis. Inceptionsfasen var den første fase, hvor gruppen diskuterede projektcasen og skabte overblik over alle moduler, hvor der blev udvalgt og sat fokus på sagsudredningsmodulet. Der blev identificeret væsentlige funktioner og der blev skabt en forståelse over forretningsdomænet.  Der blev indhentet krav på baggrund af virksomhedsmøder med EG-Team Online og undersøgelse af VUM. Disse krav blev moduleret ved at anvende software teoretiske principper. Fordelen ved anvendelsen af denne fase var at den skabte overblik over projektet samt at gruppen fik udarbejdet et struktureret inceptionsdokument som beskrev kravspecifikationen. Ydermere blev inceptionsdokumentet udarbejdet med henblik på at få en form for godkendelse til at fortsætte yderligere bearbejdelse på produktet. Ulemperne igennem fasen har været at der skulle udarbejdes så mange diagrammer, dermed fundet frem til at de fleste diagrammer har været unødvendige, f.eks. kunne sekvensdiagrammerne have været undladt. \\
Som en kombination af UP med par produktion i projektet har det været en væsentlig metode til at realisere projektet. Par produktion fungerede på den måde at der blev lavet et par af to gruppemedlemmer. Hvert par fik opgaver der skulle udarbejdes indbyrdes. Det vigtige element i metoden var at man støttede hinanden med ideer, forbedringer og korrektioner. Dette styrkede fagligheden samtidig med at man fik en forståelse for teorien. At arbejde to om en opgave var nok ulempen, da systemet som der blev bearbejdet, ikke var kompleks, som havde en hvis effekt på tiden. Det kunne vendes om og argumenteres for at alle er i gang med at tilegne sig erfaring og faglig viden, dermed har det været en nødvendighed. 
\subsection{Kombination af Scrum og UP} \label{sec:scrum og up}
Hele projektforløbet har været baseret på UP og dens faser med et forsøg på at kombinere Scrum metodologi. Igennem første iteration blev der brugt dele af scrum f.eks. blev der lavet en opgave hvor der skulle udarbejdes en general funktionalitet, opgaven blev delt ud i mindre bider til hvert gruppe par, med en kombination af Kanban. Der er forsøgt at anvende daglig scrum møder men dette har ikke fungeret helt, da disse møder fik en konflikt med de øvrige fag på semesteret. 
\subsection{Planlægning} \label{plan}
Generelle planlægning af elaborationsfasen har gruppen benyttet kanban og oversigt over milepælene, samt UP oversigten over faserne. Der blev udarbejdet en product backlog baseret på de væsentlige funktionelle og ikke-funktionelle krav og detaljerede brugsmønstre. (Se figur. 1)\\
Baseret på to iterationer har gruppen valgt at 1. iteration skulle være general funktionalitet og når gruppen nåede 2. iteration, skulle der implementeres GUI og database. Planlægningen skete udfra kanban board hvor der blev oprettet opgaver som hvert gruppe par blev tildelt og herefter udarbejdet opgaven. \\
Generalt har gruppen ikke fundet Scrum til at være brugbar i projektet. Dette skyldes flere årsager. Når der tales omkring rollefordelingen, så har gruppen ikke haft en bestemet product owner for projektet. Dette skyldes at gruppen ikke var et fuldendt team hvilket Scrum kræver. Der har hellere ikke været benyttet en scrum master fordi gruppen ikke har erfaring nok til at anvende teorien og generelt har der ikke været behov for det. Dermed har gruppen valgt at ikke benytte rollefordeling. Ydermere blev der udarbejdet en product backlog, men uden en product owner kunne den ikke blive opdateret løbende. Til det formål var det bedre og mere effektivt at anvende kanban metoden til at holde styr på hvad der blev udarbejdet og hvad der manglede. Som en del a iterationerne har fokus ligget på de forskellige faser og iterationer, hvor gruppen koncentrerede sig på den generelle funktionalitet i første iteration og GUI og database i anden iteration.\\
Der konkluderes at gruppen har anvendt for meget tid på at prøve at implementere Scrum i projektet end at gruppen fik et udbytte af det og få elementer af det til at have en general oversigt. Som nævnt igennem afsnittet blev der udarbejdet en product backlog men uden en product owner til at raffinere den løbende, har gruppen ikke fundet en grund til at opdatere det da det ville være spild af brugbar tid og dermed lagde fokus på udarbejdelsen af produktet. 
