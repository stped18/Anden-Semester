\chapter{Perspektivering}
Den nuværende fase som projektet befinder sig i er Elaborationsfasen (slutning af elaborationsfasen). Gruppen har arbejdet i 2. iteration med at få implementeret væsentlige brugsmønstre, som f.eks. find en sag eller finde en borger der er tilknyttet en sag, og oprettelse af en sag baseret på forskellige formular. \\
Det næste skridt i projektet er konstruktionsfasen der har til formål at udarbejde og implementere den manglende funktionalitet f.eks. behandle sag, tildeling af sag og dagbog, samtidig med at lave test, som gøre produktet klar til kunden. Dette kan gøres igennem 2 – 3 iterationer, hvor 1. iteration ville handle om at få optimeret kode i forhold til det der er i forvejen samt færdiggøre funktionaliteten. I 2. iteration af konstruktionsfasen ville der blive fokuseret på at få implementeret den resterende funktionalitet som f.eks. at afdelingslederen har mulighed for at tilde sager til sagsbehandlere. I 3. iteration lave en endelig version af systemet, med en brugergrænseflade der både er brugervenlig og funktionel og som er klar til test. Til sidst når projektet når transationsfasen er det til formål udgiver produktet til kunden, hvor så man får feedback og kan så justere og finpudse de resterende påpegninger til systemet. \\
Hvis projektet skulle bygges igen, med en realistisk tidsramme og med den viden som gruppen har tilegnet sig, så ville det gøres på følgende måde. Der ville med fordel kunne laves brugsmønstre og til det lave en overordnet kravspecifikation med væsentlige detaljerede brugsmønstre. Efterfølgende ville der blive udarbejdet et analyseklassediagram over alle de klasser og funktioner fundet i de detaljerede brugsmønstre og dermed udarbejde et designklassediagram til brug for implementering. Idet ville et component diagram også blive udarbejdet i forhold til at vise hvordan systemet er opbygget så der er bedre forståelse for forskellige komponenter. Til sidst begynder 1. iteration med at udarbejde funktionaliteten generelt og i 2. iteration udarbejde et databasedesign og database implementering og brugergrænseflade som en klar pakke til at blive sendt ind i en konstruktionsfase. I konstruktionsfasen ville produktet blive mere dybdegående udarbejdet og optimeret og lave en brugervenlig grænseflade igennem 2 – 3 iterationer og samtidig gøre den klar til levering for kunden til test. \\

