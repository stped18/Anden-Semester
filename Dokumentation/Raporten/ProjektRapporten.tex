\documentclass[12pt]{article}

\usepackage[Danish]{babel}
\usepackage{microtype}
\usepackage[a4paper, inner = 1.7cm, outer=1.7cm, top=2cm, bottom=2cm]{geometry}
\usepackage{fancyhdr}
\usepackage{graphicx}
\usepackage{wrapfig}
\usepackage{index}

\makeindex

\title{MMMI}
\author{By projektgruppe 02 }
\date{Marts 25 - 2019}
\pagestyle{fancy}

\begin{document}
\begin{titlepage}
\clearpage
\maketitle
\thispagestyle{empty}
\end{titlepage}


\section{Resumé}
Omfatter resuméet:\\
Den behandlede problemstilling - hvad blev der arbejdet med og hvorfor?\\
Fremgangsmåden - anvendte metoder - hvordan blev der arbejdet med det? \\(hvordan angreb I problemet og hvordan realiserede I løsningen (hvem, hvad, hvornår og hvorfor)
Hovedresultater og konklusioner  – hvad kom der ud af arbejdet?\\
\begin{Large}
(max  1 side)
\end{Large}
\section{Forord}
Indeholder forord hensigten med rapporten, målgruppe, forhistorie, anerkendelser, afleveringsdato samt underskrifter af alle projektdeltagere?\\
Bemærk: Projektdeltagernes aktive deltagelse i projektforløbet anerkendes gensidigt ved projektdeltagernes underskrifter i rapporten. 
\section{Indholdsfortegnelse}
\tableofcontents
\pagebreak
\section{Læsevejledning}
Er der en vejledning i, hvordan rapporten kan læses, eksempelvis i form af hvilken rækkefølge afsnittene kan læses i og hvordan sammenhængen er mellem de forskellige dele af rapporten, fx mellem produktrapport og bilag?\\
Er rapportens målgruppe beskrevet?
\section{Redaktionelt}
Beskriver redaktionelt skriveprocessen og ansvarsområder i skriveprocessen?\\
Ansvarsområder kan fx beskrives på fx følgende form:
\subsection{Ordliste}
Er der en kort beskrivelse af de fagtermer der bruges gennem rapporten?

\section{Indledning (30 til 40 sider til og med konklusion)}

Giver indledningen et overblik over projektet og baggrunden for det?\\
Giver indledningen et resume af den udleverede case?\\
Indeholder indledningen problemformuleringen, jfr. inceptiondokumentet?\\
Beskriver indledningen formålet med projektet?\\ Er formålet i overensstemmelse med hensigten med 2. semester? \\
Beskriver indledningen målene med projektet? Udtrykker målene specifikke, målbare resultater, jfr. inceptionsdokumentet? \\
Er målene i overensstemmelse med de overordnede mål for 2. semester som udtrykt i studieordningens kap. 9 og de mere specifikke mål for projektet, som udtrykt i fagbeskrivelsen for SI2-PRO?

\subsection{Problemformulering}

\subsection{Metoder}
Er metoden i det samlede projekt beskrevet?\\
Er det beskrevet hvordan UP og Scrum kombineres i projektet, samt hvilke fordele og ulemper der er ved det?\\

\subsubsection{Indledning til metoder}
Giver indledningen en introduktion til  afsnittet?\\
\subsubsection{Metoder}
Er metoden i det samlede projekt beskrevet?\\
Er det beskrevet hvordan UP og Scrum kombineres i projektet, samt hvilke fordele og ulemper der er ved det?\\
\subsubsection{Planlægning}
Er planlægningen af elaborationsfasen og de enkelte iterationer beskrevet.\\
Er backlogs beskrevet?\\
Er rollefordelingen i projektgruppen beskrevet?\\
Er ceremonierne beskrevet?\\
Er scrum-buts beskrevet?\\
Bygger planen på prioriteringen af kravene efter inceptionsfasen.\\

\section{Hovedtekst}
Hvis der indgår et teoretisk problem: Omfatter hovedteksten et teori-afsnit?\\
\subsection{Indlening til hovedafsnit}
Indeholder indledningen en overordnet introduktion til afsnittet?\\
\subsection{Overordnet kravspecifikation (resume, opdateret)}
Indeholder afsnittet et opdateret resumé  af systemafgrænsningen fra inceptionsdokumentet.
\subsection{Detaljeret kravspecifikation}
Omfatter den detaljerede kravspecifikation\\
Detaljeret brugsmønsterdiagram (hvis relevant)\\
Detaljerede brugsmønsterbeskrivelser\\
Detaljerede beskrivelser af supplerende krav Fx organiseret efter FURPS+\\
\subsection{Analyse}
Omfatter afsnittet overvejelser, beslutninger og resultater vedr.\\
Den statiske side af analysemodel\\
Den dynamiske side af analysemodel\\
\subsection{Design}
Omfatter afsnittet overvejelser, beslutninger og resultater vedr.\\
Softwarearkitektur\\
Subsystemdesign\\
Den statiske side af designmodel\\
Den dynamiske side af designmodel\\
Design af persistens\\
Databasedesign\\
\subsection{Implementering}
Omfatter afsnittet overvejelser, beslutninger og resultater vedr.\\  konvertering fra design til kode illustreret gennem udvalgte centrale eksempler, samt andre vigtige implementeringsbeslutninger
Omfatter afsnittet implementering af database.
\subsection{Test}
Omfatter afsnittet en beskrivelse af de udførte test samt resultatet af dem.\\
Er der medtaget resultater både fra iteration 1 og fra iteration 2?\\

\section{Diskussion}
Omfatter diskussionen hvad der er opnået, og hvad der ikke er opnået i projektet i forhold til det forventede som beskrevet i indledningen. \\
Hvad er styrkerne og svaghederne ved jeres resultater?\\
Kunne I have opnået bedre resultater?\\

\section{Konklusion}
Opsummerer konklusionen resultaterne og diskussionen af dem og giver det på problemformuleringen? \\
\section{Perspektivering}
Fremtidigt arbejde: Hvad ville de næste skridt i projektet være, hvis der var mere tid?\\
Refleksion: Hvordan ville I gribe projektet an, hvis I skulle starte forfra? 

\section{Litteraturliste}
Er litteratur angivet på en anerkendt form? \\
Er alle former for litteratur som bøger, artikler og hjemmesider medtaget? \\
Er der kildehenvisninger i teksten? \\
Materiale som gruppen ikke selv har fremstillet i dette projekt skal være angivet med kilde! \\
Er alle kildehenvisninger i teksten anført på samme måde? \\
Er der kildeangivelser på figurer, grafer etc. som projektgruppen ikke selv har frembragt? \\
\pagebreak
\section{Procesrapport (5 til 10 sider)}
\setcounter{page}{1}
\section{Oversigt over projektets kildekode}
Er der link til gitgub til javadoc
\section{Brugervejledning}
Er der en kortfattet brugervejledning?
\section{Projektlog}
Er der en adresse på og et link til projektloggen i Github?
\pagebreak
\section{Interne}
\pagenumbering{roman}
Er undersectionerne sat ind?\\
er kontrolskemaet udfyldt?
\subsection{Projektforslag}

\subsection{Inceptionsdokument}

\subsection{Rapportkontrolskema}
\pagebreak
\subsection{Diverse interne materialer}
\pagebreak
\clearpage
\section{Eksterne bilag}

\thispagestyle{empty}
Er der medtaget relevante eksterne bilag, dvs. materialer som gruppen ikke selv har produceret med som er nødvendige for at kunne læse rapporten?

\end{document}