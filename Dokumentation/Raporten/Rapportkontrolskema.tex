\section{Rapportkontrolskema}
\begin{center}
\begin{longtable}{|m{3.5cm}|m{10cm}|m{2.5cm}|}
\hline
\multicolumn{3}{|c|}{A. Produktrapport} \\
\hline
Kapitel & krav & opfyldt $+/-$ \\ \hline
Omslag & Indeholder omslaget projekttitel, uddannelsesinstitution, fakultet, institut, uddannelse, semester, kursuskode, projektperiode, vejleder, projektgruppe og projektdeltagere (fornavn, efternavn, sdu-email)? & \\
\hline
Titelblad & (Som omslag ekskl. evt. illustration + evt. kildehenvisning til evt. omslagsillustration. Omslaget kan udgøre både omslag og titelblad. Hvis der medtages selvstændigt titelblad, så er titelbladet rapportens første højre side) & \\
\hline
Resumé & 
Omfatter resuméet:
\begin{itemize}
\item Den behandlede problemstilling - hvad blev der arbejdet med og hvorfor?
\item Fremgangsmåden - anvendte metoder - hvordan blev der arbejdet med det? 
(hvordan angreb I problemet og hvordan realiserede I løsningen (hvem, hvad, hvornår og hvorfor))
\item Hovedresultater og konklusioner  – hvad kom der ud af arbejdet?
\end{itemize}
(max  1 side)& \\
\hline
Forord & Indeholder forord hensigten med rapporten, målgruppe, forhistorie, anerkendelser, afleveringsdato samt underskrifter af alle projektdeltagere? \newline
Bemærk: Projektdeltagernes aktive deltagelse i projektforløbet anerkendes gensidigt ved projektdeltagernes underskrifter i rapporten. & \\
\hline
Indholdsfortegnelse & Er der en samlet indholdsfortegnelse for hele projektrapporten?. (Højst to eller tre niveauer i indholdsfortegnelse) & + \\
\hline
Læsevejledning & Er der en vejledning i, hvordan rapporten kan læses, eksempelvis i form af hvilken rækkefølge afsnittene kan læses i og hvordan sammenhængen er mellem de forskellige dele af rapporten, fx mellem produktrapport og bilag? 
\newline Er rapportens målgruppe beskrevet? & \\
\hline
Redaktionelt & Beskriver redaktionelt skriveprocessen og ansvarsområder i skriveprocessen?\newline
Ansvarsområder kan fx beskrives på fx følgende form: 
\begin{tabular}{|c|c|c|c|}
\hline Afsnit & Ansvarlig & Bidrag fra & Kontrolleret af \\ \hline
Afsnit a & Person a & Person b & Person a,b,c \\ \hline
Afsnit b & Person b & Person a & Person a,b,c \\ \hline
Afsnit c & Person c & Person b & Person a,b,c \\ \hline 
\end{tabular} & \\
\hline
Ordliste & Er der en kort beskrivelse af de fagtermer der bruges gennem rapporten? & \\
\hline 
Indledning & Giver indledningen et overblik over projektet og baggrunden for det? & \\
\hline
& Giver indledningen et resume af den udleverede case? & \\
\hline
& Indeholder indledningen problemformuleringen, jfr. inceptiondokumentet? & \\
\hline
& Beskriver indledningen formålet med projektet? Er formålet i overensstemmelse med hensigten med 2. semester? & \\
\hline 
& Beskriver indledningen målene med projektet? Udtrykker målene specifikke, målbare resultater, jfr. inceptionsdokumentet? Er målene i overensstemmelse med de overordnede mål for 2. semester som udtrykt i studieordningens kap. 9 og de mere specifikke mål for projektet, som udtrykt i fagbeskrivelsen for SI2-PRO? & \\
\hline
Metoder & & \\ \hline
\begin{flushright}
1. Indledning 
\end{flushright}
& Giver indledningen en introduktion til  afsnittet? & \\
\hline
\begin{flushright}
2. Metode
\end{flushright}
&Er metoden i det samlede projekt beskrevet?\newline
Er det beskrevet hvordan UP og Scrum kombineres i projektet, samt hvilke fordele og ulemper der er ved det? & \\
\hline
\begin{flushright}
3. Planlægning
\end{flushright}
& Er planlægningen af elaborationsfasen og de enkelte iterationer beskrevet.\newline
Er backlogs beskrevet?\newline
Er rollefordelingen i projektgruppen beskrevet?\newline
Er ceremonierne beskrevet?\newline
Er scrum-buts beskrevet?\newline
Bygger planen på prioriteringen af kravene efter inceptionsfasen. & \\
\hline
hovedtekst & & \\ \hline
\begin{flushright}
1. Indledning
\end{flushright} & Indeholder indledningen en overordnet introduktion til afsnittet? &\\ \hline
\begin{flushright}
2. Overordnet kravspecifikation (resume, opdateret)
\end{flushright} & Indeholder afsnittet et opdateret resumé  af systemafgrænsningen fra inceptionsdokumentet. & \\ \hline
\begin{flushright}
3. Detaljeret kravspecifikation
\end{flushright} & 
Omfatter den detaljerede kravspecifikation \newline
\begin{enumerate}
\item Detaljeret brugsmønsterdiagram (hvis relevant)
\item Detaljerede brugsmønsterbeskrivelser 
\item Detaljerede beskrivelser af supplerende krav Fx organiseret efter FURPS+
\end{enumerate}
& \\
\hline
\begin{flushright}
4. Analyse
\end{flushright}
& Omfatter afsnittet overvejelser, beslutninger og resultater vedr. \newline
\begin{enumerate}
\item Den statiske side af analysemodel
\item Den dynamiske side af analysemodel
\end{enumerate}
& \\ \hline
\begin{flushright}
5. Design 
\end{flushright}
& Omfatter afsnittet overvejelser, beslutninger og resultater vedr. \newline
\begin{enumerate}
\item Softwarearkitektur
\item Subsystemdesign
\item Den statiske side af designmodel
\item Den dynamiske side af designmodel
\item Design af persistens
\item Databasedesign
\end{enumerate}
& \\ \hline
\begin{flushright}
6. Implementering
\end{flushright} 
& Omfatter afsnittet overvejelser, beslutninger og resultater vedr.  konvertering fra design til kode illustreret gennem udvalgte centrale eksempler, samt andre vigtige implementeringsbeslutninger\newline
Omfatter afsnittet implementering af database. & \\ 
\hline
\begin{flushright}
7. Test
\end{flushright}
& Omfatter afsnittet en beskrivelse af de udførte test samt resultatet af dem. & \\
\hline
& Er der medtaget resultater både fra iteration \# 1 og fra iteration \# 2? & \\ 
\hline
Diskussion &
Omfatter diskussionen hvad der er opnået, og hvad der ikke er opnået i projektet i forhold til det forventede som beskrevet i indledningen. \newline
Hvad er styrkerne og svaghederne ved jeres resultater? \newline
Kunne I have opnået bedre resultater?
& \\ \hline
Konklusion & Opsummerer konklusionen resultaterne og diskussionen af dem og giver det på problemformuleringen?  & \\
\hline
Perspektivering & 
Fremtidigt arbejde: Hvad ville de næste skridt i projektet være, hvis der var mere tid?\newline
Refleksion: Hvordan ville I gribe projektet an, hvis I skulle starte forfra? & \\
\hline
Litteraturliste & 
Er litteratur angivet på en anerkendt form? \newline
Er alle former for litteratur som bøger, artikler og hjemmesider medtaget? \newline
Er der kildehenvisninger i teksten? Materiale som gruppen ikke selv har fremstillet i dette projekt skal være angivet med kilde! \newline
Er alle kildehenvisninger i teksten anført på samme måde? \newline
Er der kildeangivelser på figurer, grafer etc. som projektgruppen ikke selv har frembragt? & \\
\hline
\end{longtable}
\end{center}


\begin{center}
\begin{longtable}{|m{3.5cm}|m{10cm}|m{2.5cm}|}
\hline
\multicolumn{3}{|c|}{B. Procesrapport} \\
\hline
Kapitel & krav & opfyldt $+/-$ \\ \hline
Læring og refleksion & Er der en redegørelse for læring og refleksion? & \\ \hline
Projektarbejdet & Er der en redegørelse for projektarbejdet? Er der en beskrivelse af det faktiske projektforløb, med inddragelse af væsentlige artefakter, som fx sprint backlogs, væsentlige ceremonier, som fx  daily scrum, rollevaretagelse mm? & \\ 
\hline
\begin{flushleft} 
Identifikationg, analyse og bearbejdning af problemer
\end{flushleft}
& Er der en redegørelse for identifikation, analyse og bearbejdning af problemer & \\ 
\hline
Projektforløb & & \\ \hline
\begin{flushleft} 
Formidling og kommunikation
\end{flushleft}
& Er der en redegørelse for formidling og kommunikation & \\ 
\hline
\begin{flushleft} 
Samarbejde i gruppen 
\end{flushleft}
& Er der en redegørelse for samarbejde i gruppen & \\ \hline
\begin{flushleft} 
Samarbejde med vejleder
\end{flushleft}
& Er der en redegørelse for samarbejde med vejleder & \\ \hline

\end{longtable}
\end{center}

\begin{center}
\begin{longtable}{|m{3.5cm}|m{10cm}|m{2.5cm}|}
\hline
\multicolumn{3}{|c|}{C. Kildekode} \\ \hline
Kapitel & krav & opfyldt $+/-$ \\ \hline
\begin{flushleft} 
Oversigt over projektets kildekode
\end{flushleft} 
& Er der en oversigt over projektets kildekode, fx filstruktur eller javadoc? & \\ \hline
\end{longtable}
\end{center}

\begin{center}
\begin{longtable}{|m{3.5cm}|m{10cm}|m{2.5cm}|}
\hline
\multicolumn{3}{|c|}{D. Brugervejledning} \\ \hline
Kapitel & krav & opfyldt $+/-$ \\ \hline
Brugervejledning & Er der en kortfattet brugervejledning? & \\ \hline

\end{longtable}
\end{center}

\begin{center}
\begin{longtable}{|m{3.5cm}|m{10cm}|m{2.5cm}|}
\hline
\multicolumn{3}{|c|}{E. Projektlog} \\ \hline
Kapitel & krav & opfyldt $+/-$ \\ \hline
Projektlog & Er der en adresse på og et link til projektloggen i Github & \\ \hline
\end{longtable}
\end{center}

\begin{center}
\begin{longtable}{|m{3.5cm}|m{10cm}|m{2.5cm}|}
\hline
\multicolumn{3}{|c|}{F. Interne bilag} \\ \hline
Kapitel & krav & opfyldt $+/-$ \\ \hline
Projektgrundlag & Er projektgrundlaget medtaget? & \\ \hline
Rapportkontrolskema & Er der et udfyldt rapportkontrolskemaet & \\ \hline
Andet & Er der andre relevante interne bilag, dvs. materialer produceret af gruppen selv? & \\ \hline
\end{longtable}
\end{center}

\begin{center}
\begin{longtable}{|m{3.5cm}|m{10cm}|m{2.5cm}|}
\hline
\multicolumn{3}{|c|}{G. Eksterne bilag} \\ \hline
Kapitel & krav & opfyldt $+/-$ \\ \hline
Eksterne bilag & 
Er der medtaget relevante eksterne bilag, dvs. materialer som gruppen ikke selv har produceret med som er nødvendige for at kunne læse rapporten? & \\ 
\hline
\end{longtable}
\end{center}

\begin{center}
\begin{longtable}{|m{3.5cm}|m{10cm}|m{2.5cm}|}
\hline
\multicolumn{3}{|c|}{H. Rapporttekniske elementer} \\ \hline
Kapitel & krav & opfyldt $+/-$ \\ \hline
Layout & Er der anvendt samme layout i alle kapitler? \newline
Er layout overskueligt/harmonisk? & \\ \hline
Sprog & Er rapporten skrevet i en neutral sprogtone? \newline
Er sproget let læseligt og flydende? \newline
Er der udført stavekontrol og kontrol af tegnsætning? & \\ \hline
Sidenummerering & Er der korrekt og konsistent sidenummerering i rapporten? & \\ 
\hline
Figurer/diagrammer & Er alle figurer konsekvent nummererede? \newline
Er der figurtitel og figurtekst til alle figurer? \newline
Er figurtitler og figurtekster dækkende og afklarende? \newline
Er figurerne tydelige og læsbare? \newline
Er figurerne informationsgivende og i den rette sammenhæng?
& \\ \hline
Tabeller & 
Er alle tabeller konsekvent nummererede? \newline
Er der en forklarende tabeltekst til alle tabeller? \newline
Er alle søjler og rækker forsynet med parametre? \newline
Er der enheder på alle relevante rækker og søjler? 
& \\ \hline
\begin{flushleft} 
Sporbarhed af begreber 
\end{flushleft}
& Er der en konsekvent brug af samme betegnelse for et givet begreb igennem rapporten? & \\ \hline




\end{longtable}
\end{center}