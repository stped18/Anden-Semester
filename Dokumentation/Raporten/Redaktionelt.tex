\chapter{Redaktionelt}
For dette skema vil følgende personer stå under punkter: 
\begin{itemize}
\item Aleksander D
\item Aleksander H
\item Aslak
\item Mathias
\item Steffen
\item Per
\item Alle - Indeholder alle personer 
\end{itemize}

\begin{center}
\begin{longtable}{|m{5cm}|m{4cm}|m{4cm}|m{3cm}|}
\hline
Afsnit & Ansvarlig & Bidrag fra & Kontrolleret af \\
\hline
Omslag & & & Alle\\
\hline
Titelblad & & & Alle\\
\hline
Resumé & & & Alle\\ \hline
Forord & & & Alle\\ \hline
Indholdsfortegnelse & Per & & Alle\\ \hline
Læsevejledning & Per & & Alle\\ \hline

Redaktionelt & Per & & \\ \hline
Ordliste & Aslak & & \\ \hline

Indledning & Aleksander H & & \\ \hline
Problemformulering & Aleksander H & & \\ \hline

Metoder & Aleksander D & & \\ \hline
Samlede metoder & Aleksander D & & \\ \hline

Hovedtekst & & & \\ \hline
Indledning til hovedafsnit & & & \\ \hline
Første iteration & & & \\ \hline
Overordnet kravspecifikation & & & \\ \hline
Detaljeret kravspecifikation & & & \\ \hline
Analyse & & & \\ \hline
Design & Aslak & Mathias & \\ \hline
Implementering & & & \\ \hline
Test & & & \\ \hline

Diskussion & & & \\ \hline

Konklusion & & & \\ \hline

Perspektivering & & & \\ \hline

Litteraturliste & & & \\ \hline

Procesrapport & & & \\ \hline
Læring og refleksion & & & \\ \hline
Projektarbejde & & & \\ \hline
Identifikation, analyse... & & & \\ \hline
Projektforløb & & & \\ \hline

Kildekode & & & \\ \hline
Oversigt over ... & & & \\ \hline

Brugervejledning & & & \\ \hline

Projektlog & & & \\ \hline


\end{longtable}
\end{center}

\section{Ordliste}

\begin{itemize}
\renewcommand\labelitemi{--}
\item Find sag: \\
Søger på sager i en given kommune. Viser de specifikke informationer i en søgt sag, uden mulighed for redigering. \\
Ved søgning på fuldt navn, skal der være mulighed for at vælge den rigtige borger før søgningen finder sted.
\item Behandle sag: \\
At man kan ændre informationer i den pågældende sag.
\item //Rapport: \\
Bruges i koden til at indikere metodeafsnit der skal uddybes i rapporten.
\item Sag afsluttet: \\
Sag afsluttets når borgeren har afsluttet det tildelte behandlingsforløb. \\
Hvis ydelsen bliver afvist i afgørelsen \\
Sagen bliver afsluttets.
\item Sag lukket: \\
Når der ikke aktivt bliver arbejdet på en pågældende sag, lukkes sagen.
\end{itemize}

