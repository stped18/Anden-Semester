\chapter{Redaktionelt}
For dette skema vil følgende personer stå under punkter: 
\begin{itemize}
\item Aleksander D
\item Aleksander H
\item Aslak
\item Mathias
\item Steffen
\item Per
\item Alle - Indeholder alle personer 
\end{itemize}

\begin{center}
\begin{longtable}{|m{5.8cm}|m{3.5cm}|m{3.5cm}|m{3.2cm}|}
\hline
\textbf{Afsnit} & \textbf{Ansvarlig} & \textbf{Bidrag fra} & \textbf{Kontrolleret af} \\
\hline
Omslag & Aleksander H \newline Per & & Alle\\
\hline
Titelblad & Aleksander H \newline Per & & Alle\\
\hline
I. Resumé & & & Alle\\ \hline
II. Forord & Per & & Alle\\ \hline
Indholdsfortegnelse & Per & & Alle\\ \hline
III. Læsevejledning & Per & & Alle\\ \hline

IV. Redaktionelt & Per & & Alle\\ \hline
Ordliste & Aslak & & Alle\\ \hline

1. Indledning & Aleksander H & & Alle\\ \hline
1.1 Problemformulering & Aleksander H & & Alle\\ \hline

2. Metoder & Aleksander D \newline Steffen & & Alle\\ \hline
2.1 Samlede metoder & Aleksander D \newline Steffen & Aslak & Alle\\ \hline

3. Hovedtekst & Per & & Alle\\ \hline
3.1 Introduktion & Aslak & & Alle\\ \hline
3.2 Overordnet kravspecifikation & Aleksander H \newline Per& & Alle\\ \hline
3.3 Detaljeret kravspecifikation & Aleksander H \newline Per& & Alle\\ \hline

3.4 Første iteration & Per & & Alle\\ \hline
3.4.2 Analyse Statisk & Alsak \newline Mathias & & Alle\\ \hline
3.4.2 Analyse Dynamisk & Aleksander D \newline Steffen & & Alle\\ \hline
3.4.3 Design Statisk & Aslak \newline Mathias  & & Alle\\ \hline
3.4.3 Design Dynamisk & Aleksander D \newline Steffen &  & Alle\\ \hline
3.4.4 Implementering & Aleksander H \newline Per & & Alle\\ \hline
3.4.5 Test & & & Alle\\ \hline

3.5 Anden iteration & Per & & Alle\\ \hline
3.5.2 Analyse Statisk & Alsak \newline Mathias & & Alle\\ \hline
3.5.2 Analyse Dynamisk & Aleksander D \newline Steffen & & Alle\\ \hline
3.5.3 Design Statisk & Aslak \newline Mathias  & & Alle\\ \hline
3.5.3 Design Dynamisk & Aleksander D \newline Steffen &  & Alle\\ \hline
3.5.4 Implementering & & & Alle\\ \hline
3.5.5 Test & & & Alle\\ \hline

4. Diskussion & & & Alle\\ \hline

5. Konklusion & & & Alle\\ \hline

6. Perspektivering & & & Alle\\ \hline

7. Litteraturliste & Per & Alle & Alle\\ \hline

B. Procesrapport & Per & & Alle\\ \hline
1. Læring og refleksion & Aslak \newline Mathias & & Alle\\ \hline
2. Projektarbejde &  Aslak \newline Mathias & & Alle\\ \hline
3. Projektforløb & Aleksander H \newline Per & & Alle\\ \hline

C. Kildekode & Alle & & Alle\\ \hline

D. Brugervejledning & Aleksander D \newline Steffen & & Alle\\ \hline

E. Projektlog & Alle & & Alle\\ \hline


\end{longtable}
\end{center}

\section{Ordliste}

\begin{itemize}
\renewcommand\labelitemi{--}
\item Find sag: \\
Søger på sager i en given kommune. Viser de specifikke informationer i en søgt sag, uden mulighed for redigering. \\
Ved søgning på fuldt navn, skal der være mulighed for at vælge den rigtige borger før søgningen finder sted.
\item Behandle sag: \\
At man kan ændre informationer i den pågældende sag.
\item //Rapport: \\
Bruges i koden til at indikere metodeafsnit der skal uddybes i rapporten.
\item Sag afsluttet: \\
Sag afsluttets når borgeren har afsluttet det tildelte behandlingsforløb. \\
Hvis ydelsen bliver afvist i afgørelsen \\
Sagen bliver afsluttets.
\item Sag lukket: \\
Når der ikke aktivt bliver arbejdet på en pågældende sag, lukkes sagen.
\end{itemize}

