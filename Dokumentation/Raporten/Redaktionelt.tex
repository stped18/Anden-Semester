\chapter{Redaktionelt}
For dette skema vil følgende personer stå under punkter: 
\begin{itemize}
\item Aleksander D
\item Aleksander H
\item Aslak
\item Mathias
\item Steffen
\item Per
\item Alle - Indeholder alle personer 
\end{itemize}

\begin{center}
\begin{longtable}{|m{5.8cm}|m{3.5cm}|m{3.5cm}|m{3.2cm}|}
\hline
\textbf{Afsnit} & \textbf{Ansvarlig} & \textbf{Bidrag fra} & \textbf{Kontrolleret af} \\
\hline
Omslag & Aleksander H \newline Per & & Alle\\
\hline
Titelblad & Aleksander H \newline Per & & Alle\\
\hline
I. Resumé & Per & & Alle\\ \hline
II. Forord & Per & & Alle\\ \hline
Indholdsfortegnelse & Per & & Alle\\ \hline
III. Læsevejledning & Per & & Alle\\ \hline

IV. Redaktionelt & Per & & Alle\\ \hline
Ordliste & Aslak & & Alle\\ \hline

1. Indledning & Aleksander H & & Alle\\ \hline
1.1 Problemformulering & Aleksander H & & Alle\\ \hline

2. Metoder & Aleksander D \newline Steffen & & Alle\\ \hline
2.1 Samlede metoder & Aleksander D \newline Steffen & Aslak & Alle\\ \hline

3. Hovedtekst & Per & & Alle\\ \hline
3.1 Introduktion & Aslak & & Alle\\ \hline
3.2 Overordnet kravspecifikation & Aleksander H \newline Per& & Alle\\ \hline
3.3 Detaljeret kravspecifikation & Aleksander H \newline Per& & Alle\\ \hline

3.4 Første iteration & Per & & Alle\\ \hline
3.4.2 Analyse Statisk & Aslak \newline Mathias & & Alle\\ \hline
3.4.2 Analyse Dynamisk & Aleksander D \newline Steffen & & Alle\\ \hline
3.4.3 Design Statisk & Aslak \newline Mathias  & & Alle\\ \hline
3.4.3 Design Dynamisk & Aleksander D \newline Steffen &  & Alle\\ \hline
3.4.4 Implementering & Aleksander H \newline Per & & Alle\\ \hline
3.4.5 Test & Per & & Alle\\ \hline

3.5 Anden iteration & Per & & Alle\\ \hline
3.5.2 Analyse Statisk & Aslak \newline Mathias & & Alle\\ \hline
3.5.2 Analyse Dynamisk & Aleksander D \newline Steffen & & Alle\\ \hline
3.5.3 Design Statisk & Aslak \newline Mathias  & & Alle\\ \hline
3.5.3 Design Dynamisk & Aleksander D \newline Steffen &  & Alle\\ \hline
3.5.4 Implementering & Aleksander D \newline Steffen & & Alle\\ \hline
3.5.5 Test & Aleksander \newline Steffen & & Alle\\ \hline
System Test & Aleksander H & & Alle\\ \hline

4. Diskussion & Aslak \newline Mathias& & Alle\\ \hline

5. Konklusion & Per & Mathias & Alle\\ \hline

6. Perspektivering & Aleksander D \newline Steffen& & Alle\\ \hline

7. Litteraturliste & Per & Alle & Alle\\ \hline

B. Procesrapport & Per & & Alle\\ \hline
1. Læring og refleksion & Aslak \newline Mathias & & Alle\\ \hline
2. Projektarbejde &  Aslak \newline Mathias & & Alle\\ \hline
3. Projektforløb & Aleksander H \newline Per & & Alle\\ \hline

C. Kildekode & Alle & & Alle\\ \hline

D. Brugervejledning & Aleksander D \newline Steffen & & Alle\\ \hline

E. Projektlog & Alle & & Alle\\ \hline


\end{longtable}
\end{center}
\newpage
\section{Ordliste}\label{sec:ordliste}

\begin{itemize}
\renewcommand\labelitemi{--}
\item VOP \\
Videregående Objektorienteret Programmering.
\item VUM \\
Voksen udredningsmetoden.
\item DHUV \\
Digitalisering af Handicap og Udsatte Voksne.
\item UP \\
Unified Process (metoder), forberedelsesfase (inception), etableringsfase (elaboration), konstruktionsfase (construction) og overdragelsesfase (transition).
\item Forretningsdomænet\\
Omstændighederne som omhandler opgaven.
\item Problemdomænet \\
Forholder sig til omstændighederne der kan observeres i den virkelige verden, med udgangspunkt i forretningsdomænet.
\item Afdelingsautentificering \\
godkendelse af korrekt afdeling.
\item Software Engineering \\
System Udvikling.
\item Ydelser \\
En given service, som tilbydes baseret på en paragraf.
\item Sikkerhedsforanstaltninger\\
blev udviklet for at sikre brugere ikke sletter noget uden at mene det eller henter information om noget de ikke må.
\item foreign key (fremmed nøgle) \\
Der bruges navnet på den primære nøgle, samt navnet på table navnet til at navngive fremmed nøglen\\
Table navn = "department" \\
primære nøgle = "departmentid" \\
Fremmed nøgle = "departmentdepartmentid"
\end{itemize}