\chapter{Redaktionelt}
For dette skema vil følgende personer stå under punkter: 
\begin{itemize}
\item Aleksander D
\item Aleksander H
\item Aslak
\item Mathias
\item Steffen
\item Per
\item Alle - Indeholder alle personer 
\end{itemize}

\begin{center}
\begin{longtable}{|m{5.8cm}|m{3.5cm}|m{3.5cm}|m{3.2cm}|}
\hline
\textbf{Afsnit} & \textbf{Ansvarlig} & \textbf{Bidrag fra} & \textbf{Kontrolleret af} \\
\hline
Omslag & & & Alle\\
\hline
Titelblad & & & Alle\\
\hline
I. Resumé & & & Alle\\ \hline
II. Forord & & & Alle\\ \hline
Indholdsfortegnelse & Per & & Alle\\ \hline
III. Læsevejledning & Per & & Alle\\ \hline

IV. Redaktionelt & Per & & \\ \hline
Ordliste & Aslak & & \\ \hline

1. Indledning & Aleksander H & & \\ \hline
1.1 Problemformulering & Aleksander H & & \\ \hline

2. Metoder & Aleksander D & & \\ \hline
2.1 Samlede metoder & Aleksander D & & \\ \hline

3. Hovedtekst & & & \\ \hline
3.1 Indtroduktion & & & \\ \hline
3.2 Overordnet kravspecifikation & & & \\ \hline
3.3 Detaljeret kravspecifikation & & & \\ \hline

3.4 Første iteration & & & \\ \hline
3.4.2 Analyse & & & \\ \hline
3.4.3 Design & Aslak & Mathias & \\ \hline
3.4.4 Implementering & & & \\ \hline
3.4.5 Test & & & \\ \hline

3.5 Anden iteration & & & \\ \hline
3.5.2 Analyse & & & \\ \hline
3.5.3 Design & & & \\ \hline
3.5.4 Implementering & & & \\ \hline
3.5.5 Test & & & \\ \hline

4. Diskussion & & & \\ \hline

5. Konklusion & & & \\ \hline

6. Perspektivering & & & \\ \hline

7. Litteraturliste & & & \\ \hline

B. Procesrapport & & & \\ \hline
1. Læring og refleksion & & & \\ \hline
2. Projektarbejde & & & \\ \hline
3. Identifikation, analyse... & & & \\ \hline
4. Projektforløb & & & \\ \hline

C. Kildekode & & & \\ \hline

D. Brugervejledning & & & \\ \hline

E. Projektlog & & & \\ \hline


\end{longtable}
\end{center}

\section{Ordliste}

\begin{itemize}
\renewcommand\labelitemi{--}
\item Find sag: \\
Søger på sager i en given kommune. Viser de specifikke informationer i en søgt sag, uden mulighed for redigering. \\
Ved søgning på fuldt navn, skal der være mulighed for at vælge den rigtige borger før søgningen finder sted.
\item Behandle sag: \\
At man kan ændre informationer i den pågældende sag.
\item //Rapport: \\
Bruges i koden til at indikere metodeafsnit der skal uddybes i rapporten.
\item Sag afsluttet: \\
Sag afsluttets når borgeren har afsluttet det tildelte behandlingsforløb. \\
Hvis ydelsen bliver afvist i afgørelsen \\
Sagen bliver afsluttets.
\item Sag lukket: \\
Når der ikke aktivt bliver arbejdet på en pågældende sag, lukkes sagen.
\end{itemize}

