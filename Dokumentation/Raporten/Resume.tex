\chapter{Resume}
Der blev givet en case hvor der var valget mellem tre moduler, hvor mist en skulle laves. Der blev her valgt sagsudredning. Her blev der fundet frem til et problem: Hvordan kan et program håndtere flere roller uden at de interagerer med hinandens data, samt hvilke sikkerhedsforanstaltninger ligger til grund for at beskytte data. Der blev brugt UP, hvor diagrammer blev lavet og arbejdet I iterationer. Der blev også brugt metoder som kanban og par produktion. Under forløbet blev der også prøvet at køre med noget Scrum. \\
Der blev valgt at arbejde med sagudredningen, da det er der en sag ville starte og det gav god mening for os at arbejde på den måde. Noget af det første der skete var at se hvad casen havde at byde på og så finde mere information omkring den. Dette blev fundet via VUM, som havde nogle gode linjer at følge. Dette endte med at give os nogle resultater der beskrev viser hvor gruppen havde tænkt sig at arbejde hen imod. \\
Der var funktioner som virkede, mens der var områder der stadig manglede lidt for at virke optimalt I forhold til hvordan rollerne fungeret. De sikkerhedsforanstaltninger gruppen lave for at sikre at en bruger ikke sletter noget uden at mene det eller vil have information om noget de ikke burde have er der valgt at bruge en postgres database vor alle kommandoer er prepared statements, så en bruger ikke kan skrive sql kode direkte i programmet. Det var også meningen at en bruger for tildelt en role, som så har nogle rettigheder, som bestemmer hvad en bruger må se og hvad en bruger må gøre.  \\