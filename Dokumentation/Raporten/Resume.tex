\chapter{Resume}
Der blev givet en case hvor der frit kunne prioriteres mellem tre moduler og hvor der skulle arbejdes med mindst et modul. 
Gruppen valgte at fokusere på udviklingen af modulet, sagsudredning. 
Her blev der fundet frem til flere potentielle problemstillinger, 
herunder hvordan et program kan håndtere flere roller uden at de interagerer med hinandens data, 
samt hvilke sikkerhedsforanstaltninger der skal ligge til grund for at beskytte data. 
Der blev brugt UP, hvor diagrammer blev lavet og arbejdet i iterationer. 
Der blev også brugt metoder som kanban og par produktion. 
Under forløbet blev der benyttet elementer af Scrum. 
Der blev arbejdet med sagsudredningen, da det er her en sag startes op. 
Det første der blev gjort, var at udarbejde krav på baggrund af casen, 
og derudover finde yderligere information omkring den. 
Dette blev fundet via VUM (Anon., 2019), som havde nogle gode retningslinjer at følge. 
Resultatet blev en beskrivelse af hvad gruppen ville arbejde hen imod. 
Sikkerhedsforanstaltninger blev udviklet for at sikre, at brugere ikke sletter noget uden at ville det eller henter information om noget de ikke må have adgang til. 
Målet var også at en bruger får tildelt en rolle, som så har nogle rettigheder, der bestemmer hvad brugeren har adgang til.